\section{Conservation of Mass and Momentum for Solid}
Assuming matter to behave like a continuum, fundamental physical laws like conservation of mass and conservation of momentum can be applied to derive a differential equation describing the motions of a solid. Information about the particular material of a solid is described through constitutive relations.
The differential solid equation will be stated in the Lagrangian reference system \cite{Holzapfel2000}, in the solid domain $\mathcal{S}$ as:
\begin{equation}\label{eq:Solid}
\rho_s \frac{\partial \bold{d}^2}{\partial t^2} = \nabla \cdot ( P ) + \rho_s f  \hspace{4mm} in \hspace{2mm} \mathcal{S}
\end{equation}
written in terms of the deformation $\bold{d}$, of the solid.
P is the second Piola-Kirchhoff stress tensor, its derivation is included in the appendix A1 for the sake of completeness.
Body forces are denoted as $f$, and are forces that originate outside the body and act on the mass of the body e.g. gravitational force. $\rho_s$ is the solid density, and $\frac{\partial}{\partial t^2}$ is the second time derivative. 

\begin{comment}
\subsection*{Locking}
The problem og shear locking can happen FEM computations with certain elements. 
[mek4250 Kent] - Locking occurs if  $ \lambda >> \nu $ that is, the material is nearly incompressible. The reason is that all the elements discussed in this course are poor at approximating the divergence. Locking refers to the case where the displacement is to small because the divergence term essentially lock the displacement. It is a numerical artifact not a physical feature. [Verbatum]
\end{comment}
