\section{Lagrangian physics}
I will start by providing a short introduction to Lagrangian physics for the sake of completeness. For a more detailed look see \cite{Holzapfel2000}.
\begin{center}
\includegraphics[scale=0.4]{continuum_mapping.png}
\end{center}
I define $ \hat{\mathcal{S}}$ as the initial stress free configuration of a given body, $\mathcal{S}$ as the reference and $\mathcal{S}(t)$ as the current configuration respectively.
I need to define a smooth mapping from the reference configuration to the current configuration:
\begin{equation}
\chi^s(\textbf{X},t) : \hat{\mathcal{S}} \rightarrow \mathcal{S}(t)     
\end{equation}
Where $\textbf{X}$ denotes a material point in the reference domain and $\chi^s$ denotes the mapping from the reference configuration to the current configuration. Let $d^s(\textbf{X},t)$ denote the displacement field which describes deformation on a body. I then set the mapping $\chi^s$   to be specified from the its current position plus the displacement from that position:
\begin{equation}
 \chi^s(\textbf{X},t) = \textbf{X}  + d^s(\textbf{X} ,t) 
\end{equation}
which can be written in terms of the displacement field:
\begin{equation}
 d^s(\textbf{X},t) = \chi^s(\textbf{X},t) -\textbf{X}   
\end{equation}

I can then define w(\textbf{X},t) as  the domain velocity which is the partial time derivative: 
\begin{equation}
 w(\textbf{X},t) = \frac{\partial \chi^s(\textbf{X},t)}{\partial t}   
\end{equation}

\subsection{Deformation gradient}
To describe the rate at which a body undergoes deformation I will need to define a deformation gradient.
If $d(\textbf{X},t)$ is a differentiable deformation field in a given body. I define the deformation gradient as:  
\begin{equation}
\label{eq:deformation_gradient}
F = \frac{\partial \chi^s(\textbf{x},t)}{\partial \textbf{X}} = \frac{\partial \textbf{X}  + d^s(\textbf{X} ,t) }{\partial \textbf{X}} =  I + \nabla d(\textbf{X},t) 
\end{equation}
which denotes relative change of position under deformation in a Lagrangian frame of reference. We see that when there is no motion, hence no deformation. The deformation gradient $F$ is simply the identity matrix. \newline
We also need a way to change between volumes, from the reference ($dv$) to current ($dV$) configuration. This is defined with the Jacobian, which is the determinant of the of the deformation gradient F:
\begin{equation}
J = \text{det}(F)
\end{equation}
The Jacobian is used to change between volumes ,if I assume infinitesimal line and area elements in the current $ds, dx$ and reference $dV,dX$ configurations. The volume elements $dv, dV$ can be expressed by the dot product:
\begin{equation}
 dv = ds\cdot dx = J dS dX
\end{equation}
This is used to get the Nansons formula:
\begin{equation}\label{eq:Nanson}
ds = JF^{-T}dS
\end{equation}
which holds for an arbitrary line element in different configurations. This will be useful later on when describing equations in different configurations.

\subsection{Strain}
The relative change of location between two particles is called strain. This is the fundamental quality the causes stress. \cite{Richter2016}. 
If we look at two neighboring points \textbf{X} and \textbf{Y}. I can describe \textbf{Y} with:
\begin{equation}
\textbf{Y} = \textbf{Y} + \textbf{X} - \textbf{X} = \textbf{X} + |\textbf{Y} - \textbf{X}| \frac{\textbf{Y} - \textbf{X}}{|\textbf{Y} - \textbf{X}|} = \textbf{X} + d\textbf{X}
\end{equation}

I write $d\textbf{X} = d\epsilon \textbf{a}_0$,
where $d\epsilon = |\textbf{Y} - \textbf{X}|$ is the distance between the two points and $\textbf{a}_0 = \frac{\textbf{Y} - \textbf{X}}{|\textbf{Y} - \textbf{X}|}$
A certain motion transform the points $\textbf{Y}$ and $\textbf{X}$ into the displaced positions $\textbf{x} = \chi^s(\textbf{X},t)$ and $\textbf{y} = \chi^s(\textbf{Y},t)$. Using Taylor?s expansion \textbf{y} can be expressed in terms of deformation gradient:
\begin{align}
\textbf{y} =& \chi^s(\textbf{Y},t) = \chi^s(\textbf{X} + d\epsilon \textbf{a}_0,t) \\
=& \chi^s(\textbf{X},t) + d\epsilon F \textbf{a}_0 + o(\textbf{Y}-\textbf{X}) \\
\end{align}
where $o(\textbf{Y}-\textbf{X})$ refers to the small error that tends to zero faster than $\textbf{X} - \textbf{Y}$
Next I define the $\textbf{stretch vector}$ $\lambda_{\textbf{a}0}$ : 
\begin{equation}
\lambda_{\textbf{a}0}(\textbf{X},t) = F(\textbf{X},t)\textbf{a}_0 
\end{equation}
If we look at the square of $\lambda$

\begin{align}
\lambda^2 &=  \lambda_{\textbf{a}0} \lambda_{\textbf{a}0} = F(\textbf{X},t)\textbf{a}_0 F(\textbf{X},t)\textbf{a}_0 \\
&= \textbf{a}_0 F^TF\textbf{a}_0 = \textbf{a}_0 C \textbf{a}_0
\end{align}
We have not introduced the important right Cauchy-Green tensor:
 \begin{equation}
 C = F^TF
\end{equation}
which is symmetric and positive definite $C = C^T$.  I also introduce the Green-Lagrangian strain tensor E:
\begin{equation}\label{eq:StrainTensor}
E = \frac{1}{2}(F^TF -I) 
\end{equation}
which is also symmetric since C and I are symmetric. This measures the squared length change under deformation.
		
\subsection{Stress}
Stress is the internal forces between neighboring particles. Stress is responsible for deformation and is therefore crucial in continuum mechanics. 

I introduce the Cauchy stress tensor:
\begin{equation}
 \sigma_s = \frac{1}{J} F(\lambda_s (tr E)I + 2\mu_sE) F^T
\end{equation}
Using \eqref{eq:Nanson} I get the first Piola-Kirchhoff stress tensor P:
\begin{equation}
 P = J \sigma F^{-T} 
\end{equation}
I also introduce the second Piola-Kirchhoff stress tensor S:
\begin{equation}
S = J F^{-1}\sigma F^{-T} = F^{-1} P = S^T 
\end{equation}
from this relation I can write the first Piola-Kirchhoff tensor by the second:
\begin{equation}
P = FS
\end{equation}

\section{Solid equation}
From the principles of conservation of mass and momentum, I get the solid equation stated in the Lagrangian reference system (Following the notation from \cite{Richter2016}):
\begin{equation}\label{eq:Solid}
\rho_s \frac{\partial d^2}{\partial t^2} = \nabla \cdot ( P ) + \rho_s f 
\end{equation}
where I used the first Piola-Kirchhoff stress tensor.

\subsection*{Locking}
The problem og shear locking can happen FEM computations with certain elements. 
[mek4250 Kent] - Locking occurs if  $ \lambda >> \nu $ that is, the material is nearly incompressible. The reason is that all the elements discussed in this course are poor at approximating the divergence. Locking refers to the case where the displacement is to small because the divergence term essentially lock the displacement. It is a numerical artifact not a physical feature. [Verbatum]






