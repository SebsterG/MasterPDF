\section{Conservation of linear momentum for a solid}
With matter assumed continuous, fundamental physical laws like conservation of mass and conservation of momentum can be applied to derive a differential equation describing the motions of a solid. Information about the particular material of a solid is added through constitutive relations.
The differential solid equation will be stated in the Lagrangian reference system \cite{Richter2016}, in the solid time domain $\mathcal{S}$ as:
\begin{equation}\label{eq:Solid}
\rho_s \frac{\partial \bold{d}^2}{\partial t^2} = \nabla \cdot ( P ) + \rho_s f  \hspace{4mm} in \hspace{2mm} \mathcal{S}
\end{equation}
written in terms of the deformation $\bold{d}$, of the solid.. 
P is the first Piola-Kirchhoff stress tensor. The derivation of Piola-Kirchhoff is included in the Appendix for the sake of completeness.
Body forces are denoted as $f$, and are forces that originate outside the body and act on the mass of the body.

Body forces are forces originating from sources outside of the body[12] that act on the volume (or mass) of the body. Saying that body forces are due to outside sources implies that the interaction between different parts of the body (internal forces) are manifested through the contact forces alone


\begin{comment}
\subsection*{Locking}
The problem og shear locking can happen FEM computations with certain elements. 
[mek4250 Kent] - Locking occurs if  $ \lambda >> \nu $ that is, the material is nearly incompressible. The reason is that all the elements discussed in this course are poor at approximating the divergence. Locking refers to the case where the displacement is to small because the divergence term essentially lock the displacement. It is a numerical artifact not a physical feature. [Verbatum]
\end{comment}
