\chapter*{Solid Equations}
In this chapter we will look briefly into the various parts that make up the solid equation. The solid equation is most commonly described in a Lagrangian description as it fits the solid problem. In a Lagrangian description the material particles are fixed with the grid points, this is a useful property when tracking the solid domain.

\section*{Reference domain}
\subsection*{Mapping and identites}
We will start by providing a short introduction to Lagrangian physics for the sake of completeness.
\begin{center}
\includegraphics[scale=0.4]{continuum_mapping.png}
\end{center}
We define $ \hat{\mathcal{S}}$ as the initial stress free configuration of a given body. $\mathcal{S}$ as the reference and $\mathcal{S}(t)$ as the current configuration respectively.
We need to define a smooth mapping from the reference configuration to the current configuration:
$$  \chi^s(t) : \hat{\mathcal{S}} \rightarrow \mathcal{S}(t)     $$ 

Following the notation of \cite{Holzapfel2000},where $\textbf{X}$ denote a material point in the reference domain and $\chi^s$ denotes the mapping from the reference configuration. 
$d^s(\textbf{X},t)$ denotes the displacement field and w(\textbf{X},t) is the domain velocity. we set the mapping
$$\chi^s(\textbf{X},t) = \textbf{X}  + d^s(\textbf{X} ,t)$$
where $d^s(\textbf{X} ,t)$ represents displacement field
$$  d^s(\textbf{X},t) = \chi^s(\textbf{X},t) -\textbf{X}   $$

$$  w(\textbf{X},t) = \frac{\partial \chi^s(\textbf{X},t)}{\partial t}   $$



Next we will need a function that describes the rate of deformation in the solid.
\subsection*{Deformation gradient}
When a continuum body undergoes deformation and is moved from the reference configuration to some current configuration we need a deformation gradient that describes the rate of deformation in the body. 
If $d(\textbf{X},t)$ is a differentiable deformation field in a given body. We define the deformation gradient as:  
$$F = \frac{\partial \chi}{\partial \textbf{X}} = I + \nabla d(\textbf{X},t)$$ 
which denotes relative change of position under deformation in a Lagrangian frame of reference.

A change in volume between reference ($dv$) and current ($dV$) configuration is defined as :
\begin{align*}
dv =& J dV\\
J =& \text{det}(F)\\
\end{align*}
Where J is the determinant of the deformation gradient F known as the Jacobian determinant or volume ratio. If there is no motion, that is $ F = I$ and $J=1$ , there is no change in volume. But we can also have the constraint $J=1$ with motion, but preserving the volume.\newline
If we assume infinitesimal line and area elements in the current $ds, dx$ and reference $dV,dX$ configurations. The volume elements $dv, dV$ can be expressed by the dot product:
$$ dv = ds\cdot dx = J dS dX$$
This is used to get the Nanson`s formula:
$$ ds = JF^{-T}dS$$
which holds for an arbitrary line element in different configurations.

\subsection*{Strain}
In continuum mechanics relative change of location of particles is called strain and this is the fundamental quality that causes stress in a material. [Godboka]. An import strain measure is the right Cauchy-Green tensor 
$$C = F^TF$$ 
which is symmetric and positive definite $C = C^T$.  We also introduce the Green-Lagrangian strain tensor E:
$$E = \frac{1}{2}(F^TF -I) $$
which is also symmetric since C and I are symmetric. This measures the squared length change under deformation.
		
\subsection*{Stress}
Stress is the internal forces between neighboring particles. We introduce the Cauchy stress tensor:
$$ \sigma_s = \frac{1}{J} F(\lambda_s (tr E)I + 2\mu_sE) F^T$$
Cauchys stress theorem states that if the traction vectors(force measured on surface area per unit) depends on $\bold{n}$ or $\bold{N}$ then they must be linear in $\bold{n}$ or $\bold{N}$. Giving: 
\begin{align*}
t(x,t,n) =& \sigma_s(x,t) \bold{n} \\
T(x,t,n) =& P(X,t) \bold{N} \\
\end{align*}
where t is the traction vector and T is the first Piola-Kirchhoff traction vector. $\sigma$ is the Cauchy stress tensor and P is the first Piola-Kirchhoff stress tensor. Using Nanson?s formula P can be written as:
$$ P = J \sigma F^{-T} $$
We also introduce the second Piola-Kirchhoff stress tensor S:
$$ S = J F^{-1}\sigma F^{-T} = F^{-1} P = S^T $$
from this relation we can write the first Piola-Kirchhoff tensor by the second:
$$ P = FS$$

\subsection*{Solid equation}
From the principles of conservation of mass and momentum, we get the solid equation stated in the Lagrangian reference system (Following the notation and theory from [Godboka]:
\begin{equation}
\rho_s \frac{\partial d^2}{\partial t^2} = \nabla \cdot ( P ) + \rho_s f 
\end{equation}

\subsection*{Locking}
The problem og shear locking can happen FEM computations with certain elements. 
[mek4250 Kent] - Locking occurs if  $ \lambda >> \nu $ that is, the material is nearly incompressible. The reason is that all the elements discussed in this course are poor at approximating the divergence. Locking refers to the case where the displacement is to small because the divergence term essentially lock the displacement. It is a numerical artifact not a physical feature. [Verbatum]






