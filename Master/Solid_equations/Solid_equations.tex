\section{Lagrangian physics}

\begin{center}
\includegraphics[scale=0.4]{continuum_mapping.png}
\end{center}
Let $ \hat{\mathcal{S}}$, $\mathcal{S}$, $\mathcal{S}(t)$ be the initial stress free configuration of a given body, the reference and  as the current configuration respectively.
I define a smooth mapping from the reference configuration to the current configuration:
\begin{equation}
\chi^s(\textbf{X},t) : \hat{\mathcal{S}} \rightarrow \mathcal{S}(t)     
\end{equation}
Where $\textbf{X}$ denotes a material point in the reference domain and $\chi^s$ denotes the mapping from the reference configuration to the current configuration. Let $d^s(\textbf{X},t)$ denote the displacement field which describes deformation on a body. The mapping $\chi^s$ can then be specified from its current position plus the displacement from that position:
\begin{equation} \label{eq:chi}
 \chi^s(\textbf{X},t) = \textbf{X}  + d^s(\textbf{X} ,t) 
\end{equation}
which can be written in terms of the displacement field:
\begin{equation}
 d^s(\textbf{X},t) = \chi^s(\textbf{X},t) -\textbf{X}   
\end{equation}

Let w(\textbf{X},t) be the domain velocity which is the partial time derivative of the displacement: 
\begin{equation}
 w(\textbf{X},t) = \frac{\partial \chi^s(\textbf{X},t)}{\partial t}   
\end{equation}

\subsection{Deformation gradient}
To describe the rate at which a body undergoes deformation I will need to define a deformation gradient.
Let $d(\textbf{X},t)$ be a differentiable deformation field in a given body, the deformation gradient is then:  
\begin{equation}
\label{eq:deformation_gradient}
F = \frac{\partial \chi^s(\textbf{x},t)}{\partial \textbf{X}} = \frac{\partial \textbf{X}  + d^s(\textbf{X} ,t) }{\partial \textbf{X}} =  I + \nabla d(\textbf{X},t) 
\end{equation}
which denotes relative change of position under deformation in a Lagrangian frame of reference. We can observe that when there is no deformation. The deformation gradient $F$ is simply the identity matrix. \newline

We also need a way to change between volumes, from the reference ($\int dv$) to current ($\int dV$) configuration. This is defined with the Jacobian, which is the determinant of the of the deformation gradient F:
\begin{equation}\label{eq:J}
J = \text{det}(F)
\end{equation}
The Jacobian is used to change between volumes, assuming infinitesimal line and area elements in the current $ds, dx$ and reference $dV,dX$ configurations. 

\begin{comment}
The volume elements $dv, dV$ can be expressed by the dot product:
\begin{equation}
 dv = ds\cdot dx = J dS dX
\end{equation}
This is used to get the Nansons formula:
\begin{equation}\label{eq:Nanson}
ds = JF^{-T}dS
\end{equation}
which holds for an arbitrary line element in different configurations. This will be useful later on when describing equations in different configurations.
\end{comment}

\subsection{Strain}
The relative change of location between two particles is called strain. Strain, strain rate and deformation is used to describe the relative motion of particles in a continuum. This is the fundamental quality that causes stress \cite{Richter2016}.

If we observe two neighboring points \textbf{X} and \textbf{Y}. I can describe \textbf{Y} with:
\begin{equation}
\textbf{Y} = \textbf{Y} + \textbf{X} - \textbf{X} = \textbf{X} + |\textbf{Y} - \textbf{X}| \frac{\textbf{Y} - \textbf{X}}{|\textbf{Y} - \textbf{X}|} = \textbf{X} + d\textbf{X}
\end{equation}

Let $d\textbf{X}$ be denoted by:
\begin{align}
d\textbf{X} =& d\epsilon \textbf{a}_0\\
d\epsilon =& |\textbf{Y} - \textbf{X}| \\
\textbf{a}_0 =& \frac{\textbf{Y} - \textbf{X}}{|\textbf{Y} - \textbf{X}|}
\end{align}
where $d\epsilon$ is the distance between the two points and $\textbf{a}_0$ is a unit vector 

We see now that $d\textbf{X}$ is the distance between the two points times the unit vector or direction from $\textbf{X}$ to $\textbf{Y}$.
\newline

A certain motion transform the points $\textbf{Y}$ and $\textbf{X}$ into the displaced positions $\textbf{x} = \chi^s(\textbf{X},t)$ and $\textbf{y} = \chi^s(\textbf{Y},t)$. Using Taylor?s expansion \textbf{y} can be expressed in terms of the deformation gradient:

\begin{align}
\textbf{y} =& \chi^s(\textbf{Y},t) = \chi^s(\textbf{X} + d\epsilon \textbf{a}_0,t) \\
=& \chi^s(\textbf{X},t) + d\epsilon F \textbf{a}_0 + \mathcal{O}(\textbf{Y}-\textbf{X}) 
\end{align}

where $\mathcal{O} (\textbf{Y}-\textbf{X})$ refers to the small error that tends to zero faster than $(\textbf{X} - \textbf{Y}) \rightarrow \mathcal{O}$. \newline

If I set $\textbf{x} = \chi^s(\textbf{X},t)$  It follows that:
\begin{align}
\textbf{y} - \textbf{x} =&  d\epsilon F \textbf{a}_0 + \mathcal{O}(\textbf{Y}-\textbf{X}) \\
=& F(\textbf{Y} - \textbf{X}) + \mathcal{O}(\textbf{Y}-\textbf{X}) 
\end{align}

Let the $\textbf{stretch vector}$ be $\lambda_{\textbf{a}0}$, which goes in the direction of $\textbf{a}_0$: 
\begin{equation}
\lambda_{\textbf{a}0}(\textbf{X},t) = F(\textbf{X},t)\textbf{a}_0 
\end{equation}

If we look at the square of $\lambda$:
\begin{align}
\lambda^2 &=  \lambda_{\textbf{a}0} \lambda_{\textbf{a}0} = F(\textbf{X},t)\textbf{a}_0 F(\textbf{X},t)\textbf{a}_0 \\
&= \textbf{a}_0 F^TF\textbf{a}_0 = \textbf{a}_0 C \textbf{a}_0
\end{align}

We have not introduced the important right Cauchy-Green tensor:
 \begin{equation}
 C = F^TF
\end{equation}
Since $\textbf{a}_0 $ is just a unit vector, we see that this measures the squared length of change under deformation. We see that in order to determine the stretch one needs only the direction of $\textbf{a}_0$ and the tensor C.
C is also symmetric and positive definite $C = C^T$.  I also introduce the Green-Lagrangian strain tensor E:
\begin{equation}\label{eq:StrainTensor}
E = \frac{1}{2}(F^TF -I) 
\end{equation}
which is also symmetric since C and I are symmetric. 
		
\subsection{Stress}
While strain, deformation and strain rate only describe the relative motion of particles in a given volume. Stress give us the internal forces between neighboring particles. Stress is responsible for deformation and is therefore crucial in continuum mechanics. The unit of stress is force per area.

I introduce the Cauchy stress tensor:
\begin{equation}
 \sigma_s = \frac{1}{J} F(\lambda_s (tr E)I + 2\mu_sE) F^T
\end{equation}
If we use this tensor on an area that is taking the stress tensor times the normal vector $\sigma_s \bold{n}$ we get the forces acting on that area.

Using the deformation gradient and the Jacobian, I get the first Piola-Kirchhoff stress tensor P:
\begin{equation}
 P = J \sigma F^{-T} 
\end{equation}
This is known as the \textit{Piola Transformation} and maps the tensor into a Lagrangian formulation which will be used when stating the solid equation.

I also introduce the second Piola-Kirchhoff stress tensor S:
\begin{equation}
S = J F^{-1}\sigma F^{-T} = F^{-1} P = S^T 
\end{equation}
from this relation the first Piola-Kirchhoff tensor can be expressed by the second:
\begin{equation}
P = FS
\end{equation}

\section{Solid equation}
The solid equation describes the motion of a solid. It is derived from the principles of conservation of mass and momentum. Stated in the Lagrangian reference system \cite{Richter2016}:
\begin{equation}\label{eq:Solid}
\rho_s \frac{\partial d^2}{\partial t^2} = \nabla \cdot ( P ) + \rho_s f 
\end{equation}
written in terms of the deformation $d$, where I used the first Piola-Kirchhoff stress tensor. $f$ is a force acting on the solid body. Considering the material law from the previous section. The stresses depend on the displacements $d$ which again depends on the strain. The solid equation will be finalized by stating the different boundary conditions needed to solve a solid problem.
\subsection{Solid Boundary Conditions}
The solid moves within the boundary of $ \partial \mathcal{S}$.  On the Dirichlet boundary $ \partial \mathcal{S}_D$ we impose a given value. These can be initial conditions or set to a value set to a designated spot on the domain. These initial conditions are defined for $d$ and $w$:
\begin{align}
d =& d_0 \text{   on   } \partial \mathcal{S}_D  \\
w(\textbf{X},t)_0 =& \frac{\partial d(t=0)}{\partial t}   
\end{align}

The forces on the boundaries need to equal an eventual external force $ \bold{f}$. These are are enforced on the Neumann boundaries:
$$ P\cdot \bold{n} = f \text{   on   } \partial \mathcal{S}_N    $$

\begin{comment}
\subsection*{Locking}
The problem og shear locking can happen FEM computations with certain elements. 
[mek4250 Kent] - Locking occurs if  $ \lambda >> \nu $ that is, the material is nearly incompressible. The reason is that all the elements discussed in this course are poor at approximating the divergence. Locking refers to the case where the displacement is to small because the divergence term essentially lock the displacement. It is a numerical artifact not a physical feature. [Verbatum]
\end{comment}
