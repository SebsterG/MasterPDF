\section{Solid equation}
The solid equation describes the motion of a solid. It is derived from the principles of conservation of mass and momentum. The solid equation is stated in the Lagrangian reference system \cite{Richter2016}, in the solid time domain $\mathcal{S}$ as:
\begin{equation}\label{eq:Solid}
\rho_s \frac{\partial \bold{d}^2}{\partial t^2} = \nabla \cdot ( P ) + \rho_s f  \hspace{4mm} in \hspace{2mm} \mathcal{S}
\end{equation}
written in terms of the deformation $\bold{d}$. 
P is the first Piola-Kirchhoff stress tensor. See Appendix for a detailed description of strain and stress leading to the Piola-Kirchhoff stress tensor. 
$f$ is a force acting on the solid body.


\begin{comment}
\subsection*{Locking}
The problem og shear locking can happen FEM computations with certain elements. 
[mek4250 Kent] - Locking occurs if  $ \lambda >> \nu $ that is, the material is nearly incompressible. The reason is that all the elements discussed in this course are poor at approximating the divergence. Locking refers to the case where the displacement is to small because the divergence term essentially lock the displacement. It is a numerical artifact not a physical feature. [Verbatum]
\end{comment}
