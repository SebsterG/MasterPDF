\subsection{Method of Manufactured Solution on the implementation of the Solid equation}
The first MMS is with the solid problem alone. Testing the solid equation \ref{eq:solid} with the condition $\bold{u} = \frac{\partial \bold{d}}{\partial t}$.\newline

Solutions $\hat{\bold{d}}$ and $\hat{\bold{u}}$ are manufactured to meet the requirements of MMS such as smoothness and complexity, choosing functions with sine and cosine. The derivatives does not become zero and we have time and space dependencies. 

\begin{align*}
\hat{\bold{d_e}} =& ( cos(y)sin(t) , cos(x)sin(t) )\\
\hat{\bold{u_e}} =& ( cos(y)cos(t), cos(x)cos(t) )
\end{align*}
\newline

The manufactured solutions are used to make a sourceterm $f_s$:
$$\rho_s \frac{\partial \hat{\bold{u}}}{\partial t} - \nabla \cdot ( P(\hat{\bold{d}}) ) = f_s $$
The solid variational formulation is written in weak form with test functions $\phi$ and $\psi$ as:
\begin{align}
\big(\rho_s \frac{\partial \bold{u}}{\partial t},\phi \big)_{\mathcal{\hat{S}}} + \big(P, \nabla \phi \big)_{\mathcal{\hat{S}}} &=f_s \\
\big( \bold{u}- \frac{\partial \bold{d}}{\partial t} ,\psi \big)_{\mathcal{\hat{S}}} &= 0 
\end{align}

These equations are solved together and we solve for $d$ and $u$ on a unit square domain. The number N denotes the number of spatial points in x and y direction. The functions u and d will be computed to match the source term. 
The computations ran with 10 timesteps and the error was calculated for each time step and then the mean of all the errors were used to calculate the convergence rates.

Even though there are two equations, we do not make a source for the second equation, since the solutions are made to uphold the criteria $\bold{u} = \frac{\partial \bold{d}}{\partial t}$.\newline

In the tables below we investigate convergence in space and time.
Starting with checking order of convergence in space, setting m = 1, the expected order of convergence will be 2. 

\begin{table}[H]
\centering
\caption{Method of Manufactured Solution on the implementation of the Solid equation in space with m = 1}
\label{tab:MMS_SOLID_SPACE}
\begin{tabular}{|l|l|l|l|l|l|l|}
\hline
\textbf{N} & $\Delta t$ & \textbf{m} & $E_u $ & $\bold{k_u}$ & $E_d $ & $\bold{k_d}$ \\ \hline
\textbf{4} & $1\times10^{-7}$ & \textbf{1} & 0.0068828 & \textbf{-} & $3.7855 \times 10^{-9} $ & \textbf{-} \\ \hline
\textbf{8} & $1\times10^{-7}$ & \textbf{1} & 0.0017204 & \textbf{2.0002577} & $9.4622 \times 10^{-10} $ & \textbf{2.0002577} \\ \hline
\textbf{16} & $1\times10^{-7}$ & \textbf{1} & 0.0004300 & \textbf{2.0000622} & $2.3654 \times 10^{-10} $ & \textbf{2.0000622} \\ \hline
\textbf{32} & $1\times10^{-7}$ & \textbf{1} & 0.0001075 & \textbf{2.0000154} & $5.9136 \times 10^{-11} $ & \textbf{2.0000154} \\ \hline
\textbf{64} & $1\times10^{-7}$ & \textbf{1} & 0.0000268 & \textbf{2.0000038} & $1.4783 \times 10^{-11} $ & \textbf{2.0000038} \\ \hline
\end{tabular}
\end{table}

Lastly I check convergence in time. The scheme tested is of first order, with a value of $\theta = 1$, hence a order of convergence of 1 is expected. Here i set N = 64 and the timestep is halved for each computation.

\begin{table}[H]
\centering
\caption{Method of Manufactured Solution on the implementation of the Solid equation in time}
\label{tab:MMS_SOLID_TIME}
\begin{tabular}{|l|l|l|l|l|l|}
\hline
N & $\bold{\Delta t}$ & $E_u [\times10^{-6}]$ & $\bold{k_u}$ & $E_u [\times10^{-8}]$ & $\bold{k_d}$ \\ \hline
64 & \textbf{0.1} & 0.027663 & \textbf{} & 0.0034221 & \textbf{} \\ \hline
64 & \textbf{0.05} & 0.013390 & \textbf{1.0467} & 0.0018093 & \textbf{0.9194} \\ \hline
64 & \textbf{0.025} & 0.007016 & \textbf{0.9324} & 0.0009246 & \textbf{0.9685} \\ \hline
64 & \textbf{0.0125} & 0.003645 & \textbf{0.9444} & 0.0004688 & \textbf{0.9798} \\ \hline
64 & \textbf{0.00625} & 0.001828 & \textbf{0.9957} & 0.0002414 & \textbf{0.9571} \\ \hline
\end{tabular}
\end{table}

\subsubsection*{Comments on the Solid MMS}
The table \ref{tab:MMS_SOLID_SPACE} shows a decrease in error with increased number of spatial points. The order of convergence tends toward the expected value of 2, hence increasing our confidence in the implementation of the solid equation.\newline

Table \ref{tab:MMS_SOLID_TIME} shows a order of convergence tending towards 1, which was expected. With order of convergence tests passed for both time and space for the solid equation and the condition $\bold{u} = \frac{\partial \bold{d}}{\partial t}$, the implementation of the solid equation is satisfactory. 


\subsection{MMS of Fluid equations with prescribed motion}
Starting by prescribing a motion to $ d$ and $w$ and manufacturing solutions $\hat{\bold{u}}$ and $\hat{p}$. Setting $\hat{\bold{u}} = \bold{w}$:
\begin{align*}
\hat{\bold{d}} =& ( cos(y)sin(t) , cos(x)sin(t) )\\
\hat{\bold{u}} = \hat{\bold{w}}=& ( cos(y)cos(t), cos(x)cos(t) ) \\
\hat{p} =& cos(x)cos(t)
\end{align*}

Solutions are made to uphold the criterias : 

$$ \nabla \cdot \bold{u} =0  ,\hspace{2mm} \frac{\partial \bold{d}}{\partial t} = \bold{w}  $$

Whilst testing the implementations of the fluid equations, the opportunity arises to also test the mappings between current and reference configurations.
The source term $f_f$ is hence made without mappings:

$$ \rho_f \frac{\partial \hat{\bold{u}}}{\partial t}  +  \nabla \hat{\bold{u}} (\hat{\bold{u}} - \frac{\partial \hat{\bold{d}}}{\partial t})  -  \nabla \cdot \sigma(\hat{\bold{u}}, \hat{p})_f  = f_f $$

Then $f_f$ is used and mapped to the reference configuration and computed:
$$ \rho_f J \frac{\partial \bold{u}}{\partial t} + (\nabla \bold{u})F^{-1}(\bold{u}-\frac{\partial \bold{d}}{\partial t})  + \nabla \cdot( J \hat{\sigma_f} F^{-T}) = J f_f$$

The computations are done on a unit square domain and the computations ran with 10 timesteps and the error was calculated for each time step and then the mean of all the errors was taken and used to calculate the convergence rates.

The scheme tested for convergence time is of first order, with a value of $\theta = 1$, hence a order of convergence of 1 is expected. Here i set N = 64 and the timestep is halved for each computation.

\begin{table}[H]
\centering
\caption{MMS ALE FSI u=w}
\label{tab:MMS_Flu?id_time}
\begin{tabular}{|l|l|l|l|l|l|}
\hline
\textbf{N} & $\Delta t$ & $E_u$ & $\bold{k_u}$ & $E_p$ & $\bold{k_p}$ \\ \hline
\textbf{64} & 0.1 & $5.1548 \times 10^{-5}$ & \textbf{-} & 0.008724 & \textbf{-} \\ \hline
\textbf{64} & 0.05 & $2.5369 \times 10^{-5}$ & \textbf{1.0228} & 0.004290 & \textbf{1.0240} \\ \hline
\textbf{64} & 0.025 & $1.2200 \times 10^{-5}$ & \textbf{1.0561} & 0.002058 & \textbf{1.0596} \\ \hline
\textbf{64} & 0.0125 & $0.56344 \times 10^{-5}$ & \textbf{1.1145} & 0.0009556 & \textbf{1.1068} \\ \hline
\end{tabular}
\end{table}

In testing the spatial convergence I set $m=2$, hence giving an expected order of convergence in the velocity of 3 and pressure of 2.

\begin{table}[H]
\centering
\caption{MMS ALE FSI u=w}
\label{tab:MMS_Fluid_space}
\begin{tabular}{|l|l|l|l|l|l|l|}
\hline
\textbf{N}  & $\Delta t$ & \textbf{m} & $E_u$                   & \textbf{$k_u$}  & $E_p$   & \textbf{$k_p$}  \\ \hline
\textbf{2}  & $1 \times 10^{-6}$ & \textbf{2} & $8.6955 \times 10^{-4}$ & \textbf{-}      & 0.01943 & \textbf{-}      \\ \hline
\textbf{4}  & $1 \times 10^{-6}$ & \textbf{2} & $1.0844 \times 10^{-4}$ & \textbf{3.0032} & 0.00481 & \textbf{2.0140} \\ \hline
\textbf{8}  & $1 \times 10^{-6}$ & \textbf{2} & $0.1354 \times 10^{-4}$ & \textbf{3.0007} & 0.00119 & \textbf{2.0120} \\ \hline
\textbf{16} & $1 \times 10^{-6}$ & \textbf{2} & $0.0169 \times 10^{-4}$ & \textbf{3.0001} & 0.00029 & \textbf{2.0074} \\ \hline
\end{tabular}
\end{table}

\subsubsection*{Comments on the Solid MMS}
In table \ref{tab:MMS_Fluid_time} the order of convergence converges as expected towards 1. In \ref{tab:MMS_Fluid_space} the MMS testing convergence in space converges toward 3 for velocity and 2 for pressure as was expected when setting $m=2$. Both results helps in building confidence that the implementation of the mapped fluid equation is correct.





