\chapter{Verification and Validation of the Fluid Structure Interaction Implementation.}
The general approach when solving a real world problem with numerical computing, starts by defining the mathematics, implementing the equations numerically and solve the equations on a computer. 

The produced solutions are used to extract data of interest. A question then immediately arises, is the solution and the extracted data trustworthy.

To answer this question we need to answer another question, are the equations solved correct numerically, if so is the problem defined correct mathematically.
Without answering these questions, being confident that your solutions are correct is difficult \cite{Selin2014}. This process of generating evidence that computed solutions meets certain requirements to fulfill an intended purpose, in the context of scientific computing, is known as Verification and Validation. The goal of this section will hence be to verify and validate the different numerical schemes outlined in the two previous chapters.  \\

The chapter starts with the process of Verification where the fluid and structure parts of the code will be verified. Following will be Validation of the code where I implement at a well known benchmark testing the fluid and structure parts individually and as a full FSI problem. After that I investigate the impact of using different time order schemes, and different lifting operator.\newline

\begin{comment}
We start with Verification, which is the process of assessing numerical correctness and accuracy of a computed solution. Then comes Validation, which is assessing physical accuracy of the numerical model, a process which is done by comparing numerical simulation with experimental data. In simple terms we check that we are solving the equations right and then that we are solving the right equations. The process of Verification has to always come before Validation. Because there is no need in checking if we are using the right equations if the equations are not solved right. 
\end{comment}

\section{Verification}
Verification, in the context of scientific computing, is the process of determining wether or not the implementation of numerical algorithms in computer code, is done correctly. \cite{Oberkampf2010}. 
In verification, we get evidence that the numerical model derived from mathematics is solved correctly by the computer. The strategy is to identify, quantify and reduce errors cause by mapping a mathematical model to a computational model. Verification does not address wether or not the computed solutions are in alignment with physics in the real world. It only tells us that our model is computed correctly or not. To verify that we are computing correctly we can compare our computed solution to an exact solution. But the problem is that there are no known exact solutions to, for instance, the Navier-Stokes equations, other than for very simplified problems. \newline

In Verification there are multiple classes of test that can be performed, and the most rigorous is the \textit{Method of Manufactured Solution}(MMS) \cite{Oberkampf2010}. Rather than looking for an exact solution, we manufacture one. The idea is to make a solution \textit{a priori},  and use this solution to generate an analytical source term for the governing PDEs and than run the PDE with the source term to get a solution hopefully matching the manufactured one. The manufactured solution does not need to have a physically realistic relation, since the solution deals only with the mathematics. The overall idea is to make solutions in a way that all terms of the given PDE are tested.
The procedure of MMS is as follows \cite{Oberkampf2010}:
\begin{itemize}
\item We define a mathematical model on the form $ L(\bold{u}) = 0$ where $L(\bold{u})$ is a differential operator and $u$ is a dependent variable.
\item Define the analytical form of the manufactured solution $\hat{\bold{u}}$
\item Use the model $L(u)$ with $\hat{\bold{u}}$ inserted to obtain an analytical source term $ f = L(\hat{\bold{u}}) $
\item Initial and boundary conditions are enforced from $\hat{\bold{u}}$
\item Then use this source term to calculate the solution $\bold{u}$, $L(\bold{u}) = f $
\end{itemize}

After the solution has been computed we perform systematic convergence tests \cite{Roache2002}. The idea behind order of convergence tests is based on the behavior of the error between the manufactured exact solution and the computed solution. 
If we let $\bold{u}$ be the numerical solution and $\hat{\bold{u}}$ be the exact solution, $|| . ||$ be the $L^2$ norm, we define the error as:

\begin{equation}
E = || \bold{u} - \hat{\bold{u}} ||
\end{equation}

When we decrease the node spacing ($ \Delta x, \Delta y$ or $ \Delta z$) or decrease timestep size($\Delta t$), we expect the solution to convergence towards a given solution and hence the error to get smaller. It is the rate of this error that lets us know wether the solution is converging correctly.
If we assume that the number of spatial points are equal in all directions the error is expressed as
\begin{equation}
\label{eq:Error}
 E = C_1 \Delta x^k+ C_2 \Delta t^l 
\end{equation}

where $ k = m+1 $ and m is the polynomial degree of the spatial elements. The error is hence dependent on the number of spatial points and the timestep.
If instance $\Delta t$ is reduced significantly, $\Delta x$ will dominate, and $\Delta t$ will be negligible. If we then look at two error where $E_{n+1}$ has finer mesh than $E_n$, using \eqref{eq:Error}:
\begin{align}
\frac{E_{n+1}}{E_n} = \big( \frac{\Delta x_{n+1}}{\Delta x_n} \big)^k \\
k = \frac{log( \frac{E_{n+1}}{E_n}) }{ log(\frac{\Delta x_{n+1}}{\Delta x_n})}
\end{align}

$k$ can is used to find the observed order of convergence and match with the theoretical order of convergence for each given problem.\newline

The manufactured solutions should be chosen to be non-trivial and analytic \cite{Oberkampf2010, Roache2002}. The solutions should be manufactured so that no derivatives vanish. For this reason trigonometric and exponential functions can be a smart choice, since they are smooth and infinitely differentiable. In short, a good manufactured solution is one that is complex enough so that it rigorously tests each part of the equations.\newline 

Starting with the verification of the solid part of the code with given displacement and velocity. Then verifying the fluid part with a given displacement, also testing the mappings between configurations. 
To do a full verification of the entire FSI problem, one needs to take into account the condition of continuity of velocity on the interface \cite{Etienne2006}, the stresses need to equal on the interface and the flow needs to be divergence free. Manufacturing such a solution is very difficult \cite{Etienne2012}. The author has yet to find a paper that manufactures a solution fulfilling all the condition in a rigorous manner.
MMS of full FSI is therefore  out the scope of this thesis. 

\subsection{Method of Manufactured Solution on the implementation of the Solid equation}
The MMS test is constructed to verify the implementation of the solid equation \eqref{eq:solid}, with the restriction $\bold{u}=\frac{\partial d}{\partial t}$.

Solutions $\hat{\bold{d}}$ and $\hat{\bold{u}}$ are manufactured with sine and cosine such that the derivatives does not become zero and we have temporal and spatial dependencies.
The solutions are also manufactured to uphold the restriction $\bold{u}=\frac{\partial d}{\partial t}$.
\begin{align*}
\hat{\bold{d_e}} =& ( cos(y)sin(t) , cos(x)sin(t) )\\
\hat{\bold{u_e}} =& ( cos(y)cos(t), cos(x)cos(t) )
\end{align*}
The manufactured solutions are used to produce a sourceterm $f_s$ :
\begin{equation}
\rho_s \frac{\partial \hat{\bold{u_e}}}{\partial t} - \nabla \cdot ( P(\hat{\bold{d_e}}) ) = f_s 
\end{equation}

The equations are solved for $\bold{d}$ and $\bold{u}$ on a unit square domain. The number N denotes the number of spatial points in x and y direction. The functions u and d will be computed to match the source term $f_s$. 
The computations were simulated for 10 time steps and the error was calculated for each time step and then the mean of all the errors were used as a measure of the error.
\begin{table}[H]
\centering
\caption{Method of Manufactured Solution on the implementation of the Solid equation in space with m = 1}
\label{tab:MMS_SOLID_SPACE}
\begin{tabular}{|l|l|l|l|l|l|l|}
\hline
\textbf{N} & $\Delta t$ & \textbf{m} & $E_u $ & $\bold{k_u}$ & $E_d $ & $\bold{k_d}$ \\ \hline
\textbf{4} & $1\times10^{-7}$ & \textbf{1} & 0.0068828 & \textbf{-} & $3.7855 \times 10^{-9} $ & \textbf{-} \\ \hline
\textbf{8} & $1\times10^{-7}$ & \textbf{1} & 0.0017204 & \textbf{2.0002577} & $9.4622 \times 10^{-10} $ & \textbf{2.0002577} \\ \hline
\textbf{16} & $1\times10^{-7}$ & \textbf{1} & 0.0004300 & \textbf{2.0000622} & $2.3654 \times 10^{-10} $ & \textbf{2.0000622} \\ \hline
\textbf{32} & $1\times10^{-7}$ & \textbf{1} & 0.0001075 & \textbf{2.0000154} & $5.9136 \times 10^{-11} $ & \textbf{2.0000154} \\ \hline
\textbf{64} & $1\times10^{-7}$ & \textbf{1} & 0.0000268 & \textbf{2.0000038} & $1.4783 \times 10^{-11} $ & \textbf{2.0000038} \\ \hline
\end{tabular}
\end{table}

In table \ref{tab:MMS_SOLID_SPACE} we set m = 1, and vary the number of spatial points from 4 to 64 keeping $\Delta t = 10^{-7}$. The error $E_u$ and $E_d$ gets smaller for increasing values of N.The order of convergence $k_u$ and $k_d$ converges toward the expected value of 2.

\begin{table}[H]
\centering
\caption{Method of Manufactured Solution on the implementation of the Solid equation in time}
\label{tab:MMS_SOLID_TIME}
\begin{tabular}{|l|l|l|l|l|l|}
\hline
N & $\bold{\Delta t}$ & $E_u [\times10^{-6}]$ & $\bold{k_u}$ & $E_u [\times10^{-8}]$ & $\bold{k_d}$ \\ \hline
64 & \textbf{0.1} & 0.027663 & \textbf{} & 0.0034221 & \textbf{} \\ \hline
64 & \textbf{0.05} & 0.013390 & \textbf{1.0467} & 0.0018093 & \textbf{0.9194} \\ \hline
64 & \textbf{0.025} & 0.007016 & \textbf{0.9324} & 0.0009246 & \textbf{0.9685} \\ \hline
64 & \textbf{0.0125} & 0.003645 & \textbf{0.9444} & 0.0004688 & \textbf{0.9798} \\ \hline
64 & \textbf{0.00625} & 0.001828 & \textbf{0.9957} & 0.0002414 & \textbf{0.9571} \\ \hline
\end{tabular}
\end{table}

In table \ref{tab:MMS_SOLID_TIME} we check the temporal convergence. The number of spatial points has been fixed to 64, with varying $\Delta t $ from $0.1$ halving each step to $0.0065$. The error $E_u$ and $E_d$ gets smaller for decreasing values of $\Delta t$. The scheme tested is temporal first order accurate, by setting a value $\theta = 1$ expecting a order of convergence in temporal direction of 1. In table \ref{tab:MMS_SOLID_TIME} convergence $k_u$ and $k_d$ tends towards 1.

\subsection{MMS of Fluid equations with prescribed motion}
In this section we verify the fluid equations \eqref{eq:NS_mapped} in the ALE framework computed from a reference domain, with a prescribed motion.

The functions $\hat{\bold{u}}$, $\hat{\bold{d}}$ and $\hat{p}$ are manufactured to uphold the restriction \eqref{eq:restriction} and incompressible fluid \eqref{}, and are made with sine and cosine function to uphold the criteria of MMS. The fluid and domain velocity are set to be equal: $\hat{\bold{u}} = \bold{w}$. 
\begin{align*}
\hat{\bold{d}} =& ( cos(y)sin(t) , cos(x)sin(t) )\\
\hat{\bold{u}} = \hat{\bold{w}}=& ( cos(y)cos(t), cos(x)cos(t) ) \\
\hat{p} =& cos(x)cos(t)
\end{align*}

Whilst testing the implementations of the fluid equations, the opportunity arises to also test the mappings between current and reference configurations.
The source term $f_f$ is produced without mappings:

$$ \rho_f \frac{\partial \hat{\bold{u}}}{\partial t}  +  \nabla \hat{\bold{u}} (\hat{\bold{u}} - \frac{\partial \hat{\bold{d}}}{\partial t})  -  \nabla \cdot \sigma(\hat{\bold{u}}, \hat{p})_f  = f_f $$

To be specific, we use $f_f$ from the current configuration and map it to the reference:
$$ \rho_f J \frac{\partial \bold{u}}{\partial t} + (\nabla \bold{u})F^{-1}(\bold{u}-\frac{\partial \bold{d}}{\partial t})  + \nabla \cdot( J \hat{\sigma_f} F^{-T}) = J f_f$$

The computations are performed on a unit square domain and the computations were simulated with 10 timesteps and the error was calculated for each time step and then the mean of all the errors was taken and used as a measure of the error.

\begin{table}[H]
\centering
\caption{Results for Method of Manufacture Solutions test for fluid equations}
\label{tab:MMS_Flu?id_time}
\begin{tabular}{|l|l|l|l|l|l|}
\hline
\textbf{N} & $\Delta t$ & $E_u$ & $\bold{k_u}$ & $E_p$ & $\bold{k_p}$ \\ \hline
\textbf{64} & 0.1 & $5.1548 \times 10^{-5}$ & \textbf{-} & 0.008724 & \textbf{-} \\ \hline
\textbf{64} & 0.05 & $2.5369 \times 10^{-5}$ & \textbf{1.0228} & 0.004290 & \textbf{1.0240} \\ \hline
\textbf{64} & 0.025 & $1.2200 \times 10^{-5}$ & \textbf{1.0561} & 0.002058 & \textbf{1.0596} \\ \hline
\textbf{64} & 0.0125 & $0.56344 \times 10^{-5}$ & \textbf{1.1145} & 0.0009556 & \textbf{1.1068} \\ \hline
\end{tabular}
\end{table}

In table \ref{tab:MMS_Flu?id_time} we check the temporal convergence keeping the spatial points constant with $N=64$. Decreasing $\Delta t$ by half, from 0.1 to 0.0125. The errors for fluid velocity and pressure $E_u$ and $E_p$ decrease with decreasing time steps. The compute convergence $k_u$ and $k_p$ tends toward a value of 1.

\begin{table}[H]
\centering
\caption{Results of MMS ALE FSI u=w}
\label{tab:MMS_Fluid_space}
\begin{tabular}{|l|l|l|l|l|l|l|}
\hline
\textbf{N}  & $\Delta t$ & \textbf{m} & $E_u$                   & \textbf{$k_u$}  & $E_p$   & \textbf{$k_p$}  \\ \hline
\textbf{2}  & $1 \times 10^{-6}$ & \textbf{2} & $8.6955 \times 10^{-4}$ & \textbf{-}      & 0.01943 & \textbf{-}      \\ \hline
\textbf{4}  & $1 \times 10^{-6}$ & \textbf{2} & $1.0844 \times 10^{-4}$ & \textbf{3.0032} & 0.00481 & \textbf{2.0140} \\ \hline
\textbf{8}  & $1 \times 10^{-6}$ & \textbf{2} & $0.1354 \times 10^{-4}$ & \textbf{3.0007} & 0.00119 & \textbf{2.0120} \\ \hline
\textbf{16} & $1 \times 10^{-6}$ & \textbf{2} & $0.0169 \times 10^{-4}$ & \textbf{3.0001} & 0.00029 & \textbf{2.0074} \\ \hline
\end{tabular}
\end{table}

In table \ref{tab:MMS_Fluid_space} we check the spatial convergence keeping the time step constant as $\Delta t=10^{-6}$. Increasing spatial points N from 2 to 16. The error $E_u and E_p$ decreases with increasing spatial points. Computed convergence $k_u$ tends toward 3 and $k_p$ tends towards 2.

\subsubsection*{Discussion of the MMS tests}
The MMS test of the solid equation has a clear trend toward 2 in spatial direction, and 1 in temporal direction. The temporal convergence rate k is not exactly 1, and could be because of the number of spatial points $N=64$ is not high enough. With this in mind the trends shows convergence towards the theoretical convergence, which concludes that the solid equation has been implemented correctly.

The MMS test of the fluid equations computed from the reference domain shows trends in the spatial convergence toward 3 in the fluid velocity and 2 in pressure, which is expected. For the temporal convergence of the fluid equation the trend is towards 1 but is not exactly 1. The reason could be that the number of spatial points are not high enough, and also that the fluid equations have been computed on a reference domain. Nonetheless the convergence rates are sufficient and giving the conclusion that the fluid equations are implemented correctly.

It should be noted that a more rigorous MMS test of the FSI problem would be to test the entire FSI problem, and not splitting the test into parts. To do a full MMS of the entire FSI problem, one needs to take into account the condition of continuity of velocity on the interface \cite{Etienne2006}, the stresses need to equal on the interface and the flow needs to be divergence free. Manufacturing such a solution is very difficult \cite{Etienne2012}. The author has yet to find a paper that manufactures a solution fulfilling all the condition in a rigorous manner. For this reason the MMS was split into parts, and for the intended use the author finds the results from MMS tests sufficient .


\newpage

\section{Validation}
After the code has been verified, we move on to Validation which is the process of determining if a model gives an accurate representation of real world physics within the bounds of the intended use \cite{Selin2014}. A model is made for a specific purpose, its only valid with respect to that purpose \cite{Macal2005}. If the purpose is complex and trying to answer multiple questions, then the validity needs to be determined for each question. The idea is to validate the solver, \textsl{brick by brick}, Starting with simple testing of each part of the model and build more complexity and eventually test the whole model.\newline

Three issues have been identified in this process \cite{Selin2014}: Quantifying the accuracy of the model by comparing responses with experimental responses, interpolation of the model to conditions corresponding to the intended use and determining the accuracy of the model for the conditions under which its meant to be used. Well known benchmarks will be used as reference points and these tests supply us with a problem setup, initial and boundary conditions, and lastly results that we can compare with. \newline

The process of Validation is also, as I have experienced, a way to figure out at what size timestep and number of spatial points the model can handle. As we will see in the chapter all the benchmarks are run with different timesteps and number of cells to see how the model reacts. The problem with using benchmarks with known data for comparison is that we do not test the model blindly. It is easier to mold the model to the data we already have. As Oberkampf and Trucano in \cite{Selin2014} puts it "Knowing the "correct" answer beforehand is extremely seductive, even to a saint.''. Knowing the limitations of our tests will therefore strengthen our confidence in the model. It really can be an endless process of verifying and validating if one does not clearly know the bounds of sufficient accuracy. \cite{Selin2014} \\


The following tests are done with the solid and fluid alone. Testing both steady and unsteady cases. Lastly the full FSI model is tested with steady and unsteady cases.

\subsection{Fluid-Structure Interaction between an elastic object and laminar incompressible flow} \label{sec:HronTurek}
This benchmark is based on the older CFD benchmark \say{Benchmark Computations of Laminar Flow Around a Cylinder} \cite{Turek1996}. The benchmark is called \say{Proposal for numerical benchmarking of fluid-structure interaction between an elastic object and laminar incompressible flow}, and provides a proposal for a FSI benchmark \cite{Hron2006a}. The authors provide a problem setup and results from their implementation, and opens for others to also contribute data. \newline

The difference from the older benchmark is that now there has been added an elastic \say{flag} behind the cylinder in the flow direction. The benchmark provides a computational domain, and boundary conditions. It provides 9 subproblems in 3 parts split up into: a solid mechanical part named CSM(1,2 and 3), fluid mechanical part named CFD(1,2 and 3) and a full FSI part named FSI(1,2 and 3). The chapter starts by defining the computing domain, the boundary conditions and quantities for comparison. We then split up into the three parts, CSM, CFD and FSI, listing how the subtests are computed and providing results.

\subsection*{Problem Defintion}
\subsubsection*{Domain}
\begin{center}
\includegraphics[scale=0.4]{./Verification_Validation/Hron_Turek/Domain_drawing.png}
\end{center}

The computational domain consists of a circle with an elastic bar behind the circle. The circle is positioned at (0.2, 0.2) making it 0.05 of center from bottom to top. This shift in the domain is done to induce oscillations to an otherwise steady flow.
\begin{table}[H]
\centering
\caption{Domain parameters}
\label{my-label}
\begin{tabular}{|l|l|l|l|l|}
\hline
L & H & l & h & A \\ \hline
2.5 & 0.41 & 0.35 & 0.02 & (0.2, 0.6) \\ \hline
\end{tabular}
\end{table}

\subsubsection*{Boundary conditions}
A parabolic profile has been prescribed to the inlet velocity that increases from $t=[0,2]$ and is kept constant after $t = 2$.
The fluid velocity on upper and lower walls are set to zero, normally called a \say{no slip} condition.

\begin{align*}
u(0,y) &= 1.5u_0 \frac{y(H-y)}{(\frac{H}{2})^2}  \\
u(0,y,t) &= u(0,y)\frac{1-cos(\frac{\pi}{2}t)}{2} \text{  for  } t<2.0 \\
u(0,y,t) &= u(0,y) \text{  for  } t \leq 2.0
\end{align*}

\subsubsection*{Quantities for comparison}
When the fluid moves around the circle and bar it exerts a frictional force. These forces are split into drag and lift and calculated as follows:
$$ (F_d, F_L) = \int_S \sigma_f n dS $$ 
where S is the part of the circle and bar in contact with the fluid. \\
We set a point $A = (0.2,0.6)$ on the right side of the bar. Where this point is in the spatial direction gives a value for how much the bar has deformed. \\
For some given inflow conditions, unsteady solutions appear. For the unsteady solutions the values, meaning drag and lift, and displacement in x and y directions, are represented by the mean and amplitude values:
\begin{align}
mean =& \frac{1}{2} (max + min) \\
amplitude =& \frac{1}{2} (max - min)\\
\end{align}

In each test the values denoted as \textbf{ref} are the values taken from the original benchmark proposal paper \cite{Hron2006a}. Elements are the computational elements in the mesh and dofs are the \textit{degrees of freedom} often simply called the number of unknowns.

\subsubsection{CFD test cases}
The CFD tests can be simulated in two ways. The first way assumes the bar to be rigid object, meaning that the computational domain is just the fluid domain, and a no slip condition has been set on the interface. The other way, which is implemented in this thesis, is by computing the problem as a full FSI problem by setting $\rho_s=10^{6}$ and $\mu_s=10^{12}$, such that the bar is almost completely rigid, only giving rise to very small deformation (in the $10^{-9}-10^{-10}$ range).

The CFD tests cases, CFD1 and CFD2  are simulated with Reynolds numbers 20 and 100 converging to steady solutions. The CFD3 test case has a Reynolds number 200 which will induce oscillations behind the circle and bar, giving fluctuations in the fluid velocity, hence giving unsteady solutions.

\begin{figure}[H]
\centering
\includegraphics[scale=0.40,trim={20mm 34mm 20mm 30mm},clip]{./Verification_Validation/Hron_Turek/FSI_domain_b2.png}
\caption{Picture of entire FSI computing domain with 11556 cells}
\label{fig:fullmesh}
\end{figure}
\begin{figure}[H]
\centering
\includegraphics[scale=0.362,trim={0mm 0mm 0mm 0mm},clip]{./Verification_Validation/Hron_Turek/FSI_domain_b2_zoom.png}
\caption{Picture of FSI computing domain with 11556 cells, zoomed in with the solid domain marked in pink. Around the circle we can see a small boundary layer with the width of 2 cells}
\label{fig:partmesh}
\end{figure}

\vspace{1cm}

\begin{table}[H]
\centering
\caption{Summary of all the parameters in CFD tests}
\label{my-label}
\begin{tabular}{|l|l|l|l|}
\hline
Parameters & CFD1 & CFD2 & CFD3 \\ \hline
$\rho_f [10^3 \frac{kg}{m^3}]$ & 1 & 1 & 1 \\ \hline
$\nu_f [10^{-3} \frac{m^2}{s}]$ & 1 & 1 & 1 \\ \hline
$ U [\frac{m}{s}] $ & 0.2 & 1 & 2 \\ \hline
Re = $\frac{Ud}{\nu_f}$ & 20 & 100 & 200 \\ \hline
\end{tabular}
\end{table}

\begin{table}[H]
\centering
\caption{Results of CFD1 case run as full FSI, with almost rigid bar}
\label{tab:CFD1}
\begin{tabular}{|l|l|l|l|}
\hline
\textbf{elements} & \textbf{dofs} & \textbf{Drag} & \textbf{Lift} \\ \hline
2474 & 21749 & 14.059 & 1.100 \\ \hline
7307 & 63365 & 14.110 & 1.080 \\ \hline
11556 & 99810 & 14.200 & 1.1093 \\ \hline
\textbf{ref} & \textbf{} & \textbf{14.29} & \textbf{1.119} \\ \hline
\end{tabular}
\end{table}

Table \ref{tab:CFD1} shows the results for the CFD1 testcase, showing convergence towards the referential values in Drag and Lift for increasing elements and dofs.

\begin{table}[H]
\centering
\caption{Results of CFD2 case run as full FSI with almost rigid bar}
\label{tab:CFD2}
\begin{tabular}{|l|l|l|l|}
\hline
\textbf{elements} & \textbf{dofs} & \textbf{Drag} & \textbf{Lift} \\ \hline
2474 & 21749 & 134.9 & 10.38 \\ \hline
7307 & 63365 & 135.4 & 10.0 \\ \hline
11556 & 99810 & 136.1 & 10.41 \\ \hline
\textbf{ref} & \textbf{} & \textbf{136.7} & \textbf{10.53} \\ \hline
\end{tabular}
\end{table}

Table \ref{tab:CFD2} shows the results for CFD2 tending towards the referential values in Drag and Lift for increasing elements and dofs.

\begin{figure}[H]
\label{fig:CFD2}
\includegraphics[scale=0.45, trim={9mm 0mm 0mm 10mm},clip]{./Verification_Validation/Hron_Turek/CFD2.png}
\caption{CFD2 steady state case fluid flow with 11556 cells}
\end{figure}

In the CFD2 case notice, in figure \ref{fig:CFD2}, that since the circle and bar are positioned non symmetric in the y direction there is more fluid flowing closer to the upper boundary. This non symmetry is what induces oscillations when the fluid inlet velocity is higher. The CFD2 case has an inlet velocity just below the point of inducing oscillations.

\begin{table}[H]
\centering
\caption{Results of unsteady state case CFD 3 with $\Delta t = 0.01$}
\label{CFD3_dt001}
\begin{tabular}{|l|l|l|l|}
\hline
\textbf{elements} & \textbf{dofs} & \textbf{Drag} & \textbf{Lift} \\ \hline
2474 & 21749 & $434.42 \pm 4.28$ & $-15.63 \pm 407.59$ \\ \hline
7307 & 63365 & $435.54 \pm 5.06$ & $-11.77 \pm 425.73$ \\ \hline
11556 & 99810 & $438.13 \pm 5.42$ & $ -10.01 \pm 435.40 $ \\ \hline
ref & ref & $\bold{439.45} \pm \bold{5.61 }$ & $\bold{ -11.893} \pm \bold{437.81 }$ \\ \hline
\end{tabular}
\end{table}

Table \ref{CFD3_dt001} shows the results for the CFD3 testcase with $\Delta t = 0.001$, the results show clear convergence toward the \textbf{ref} value for increased number of cells and dofs.

\subsubsection{Discussion of CFD results}
The CFD1 and CFD2 steady solution test cases gives satisfactory results compared to the referential values given in the benchmark paper.

\subsubsection{CSM test cases}
The CSM tests are calculated using only the bar as computing domain. A body force $f_s$ is set a gravitational force $g$, which has been kept fixed throughout the CSM tests, changing only the material parameters of the solid. The tests CSM1 and CSM2 gives rise to steady state solutions. The difference between them is a more slender bar. The CSM 3 gives unsteady solutions and even more slender causing the bar, if modeled correctly to down and back up. Since there is no friction from the movement in the model it should,if energy is preserved, make the down to a point and up to its initial state, and repeat this motion. 

\begin{center}
\begin{figure}[H]
\caption{Picture of the coarsest solid mesh used in the MMS test}
\includegraphics[scale=0.50,trim={18mm 55mm 18mm 55mm},clip]{./Verification_Validation/Hron_Turek/structure.png}
\end{figure}
\end{center}

\vspace{0cm}

\begin{table}[H]
\centering
\caption{Summary table of the parameters used in the CSM tests}
\label{my-label}
\begin{tabular}{|l|l|l|l|}
\hline
Parameters & CSM1 & CSM2 & CSM3 \\ \hline
$\rho_f[10^3 \frac{kg}{m^3}]$ & 1 & 1 & 1 \\ \hline
$\nu_f [10^{-3} \frac{m^2}{s}]$ & 1 & 1 & 1 \\ \hline
$u_0$ & 0 & 0 & 0 \\ \hline
$\rho_s[10^3 \frac{kg}{m^3}]$ & 1 & 1 & 1 \\ \hline
$\nu_s$ & 0.4 & 0.4 & 0.4 \\ \hline
$\mu_s[10^6 \frac{m^2}{s}]$ & 0.5 & 2.0 & 0.5 \\ \hline
$g $ & 2 & 2 & 2 \\ \hline
\end{tabular}
\end{table}

\begin{table}[H]
\centering
\caption{Results of the steady CSM1 case from coarse to fine mesh.}
\label{tab:CSM1}
\begin{tabular}{|l|l|l|l|}
\hline
elements & dofs & $d_x(A) [\times10^{-3}]$ & $d_y(A) [\times10^{-3}]$ \\ \hline
725 & 1756 & -5.809 & -59.47 \\ \hline
2900 & 6408 & -6.779 & -64.21 \\ \hline
11600 & 24412 & -7.085 & -65.63 \\ \hline
46400 & 95220 & -7.116 & -65.74 \\ \hline
\textbf{ref} & \textbf{ref} & \textbf{-7.187} &  \textbf{-66.10} \\ \hline
\end{tabular}
\end{table}


\begin{table}[H]
\centering
\caption{Results of the steady CSM2 case from coarse to fine mesh.}
\label{tab:CSM2}
\begin{tabular}{@{}|l|l|l|l|@{}}
\hline
Elements & Dofs & $d_x(A) [\times10^{-3}] $& $d_y(A) [\times10^{-3}] $\\ \hline
725 &  1756 & -0.375 & -15.19 \\ \hline
2900 & 6408 & -0.441 & -16.46\\ \hline
11600 & 24412 & -0.462 & -16.84 \\ \hline
46400 & 95220 & -0.464 & -16.87\\ \hline
\textbf{ref} & \textbf{ref} &  \textbf{-0.469} &  \textbf{-16.97} \\ \hline
\end{tabular}
\end{table}


\begin{table}[H]
\centering
\caption{Results of the unsteady CSM3 case with mesh from coarse to fine.}
\label{tab:CSM3}
\begin{tabular}{|l|l|l|l|}
\hline
elements & dofs & $d_x(A) [\times10^{-3}]$ & $d_y(A)[\times10^{-3}]$ \\ \hline
725 & 1756 & $-11.743 \pm 11.744$ & $-57.952 \pm 58.940$ \\ \hline
2900 & 6408 & $-13.558 \pm 13.559$ & $ -61.968 \pm  63.440 $ \\ \hline
11600 & 24412 & $ -14.128 \pm 14.127$ & $-63.216 \pm 64.744 $ \\ \hline
46400 & 95220 & $ -14.182 \pm 14.181 $ & $ -63.305 \pm 64.843 $ \\ \hline
\textbf{ref} &  & $ \textbf{-14.305} \pm \textbf{14.305} $ & $ \textbf{-63.607} \pm  \textbf{65.160} $ \\ \hline
\end{tabular}
\end{table}

The tables \ref{tab:CSM1} ,\ref{tab:CSM2} and \ref{tab:CSM2} shows the results of the CSM1, CSM2 and CSM3 cases respectively. All three show a clear tendency towards the referential values when increasing the number of elements.

\begin{figure}[H]  
\centering
  \begin{minipage}[b]{0.60\linewidth}
    \centering
    \includegraphics[width=0.9\linewidth,trim={2mm 2mm 5mm 5mm},clip]{./Verification_Validation//Hron_Turek/dis_x.png} 
    \caption{Displacement in x direction, timeinterval (0,10)} 
    \vspace{4ex}
  \end{minipage}%%
  \begin{minipage}[b]{0.60\linewidth}
    \centering
    \includegraphics[width=0.9\linewidth,trim={2mm 2mm 5mm 5mm},clip]{./Verification_Validation//Hron_Turek/dis_y.png} 
    \caption{Displacement in y direction, timeinterval (0,10)} 
    \vspace{4ex}
  \end{minipage} 
  \begin{minipage}[b]{0.60\linewidth}
    \centering
    \includegraphics[width=0.9\linewidth,trim={2mm 2mm 5mm 5mm},clip]{./Verification_Validation//Hron_Turek/dis_x_short.png} 
    \caption{Displacement in x direction, timeinterval (8,10)} 
    \vspace{4ex}
  \end{minipage}%% 
  \begin{minipage}[b]{0.60\linewidth}
    \centering
    \includegraphics[width=0.9\linewidth,trim={2mm 2mm 5mm 5mm},clip]{./Verification_Validation/Hron_Turek/dis_y_short.png} 
    \caption{Displacement in y direction, timeinterval (8,10)} 
    \vspace{4ex}
  \end{minipage} 
  \caption{Plots of the results for CSM3 showing Displacement of point A}
  \label{plot:CSM3} 
\end{figure}

The figure \ref{plot:CSM3} is of displacement at the point A in x and y direction of the CSM3 test. The CSM3 test was run with Crank-Nicholson, $\theta = 0.5$, and it can be seen that in the y displacement the bar returns to it initial state, that is with zero displacement. This results indicates that the energy in the system has been preserved.

\subsubsection*{Discussion of the CSM results}


\subsubsection*{FSI tests}
The FSI tests are run with 2 different inflows conditions. FSI1 gives a steady state solution while the others are unsteady. FSI-2 gives the largest deformation is therefore considered the most difficult of the three \cite{Richter2013}, giving deformations of 2.5 times greater than the flag height. Vinje 2016 \cite{Vinje2016} reported not being able to compute FSI-2 model due to shortcomings in the linear elasticity solid model. The FSI-3 test has the highest inflow speed giving medium deformations but more rapid oscillations.


\begin{table}[h!]
\centering
\caption{FSI Parameters}
\label{my-label}
\begin{tabular}{|l|l|l|l|}
\hline
Parameters & FSI1 & FSI2 & FSI3 \\ \hline
$\rho_f[10^3 \frac{kg}{m^3}]$ & 1 & 1 & 1 \\ \hline
$\nu_f [10^{-3} \frac{m^2}{s}]$ & 1 & 1 & 1 \\ \hline
$u_0$ & 0.2 & 1 & 2 \\ \hline
Re = $\frac{U d}{\nu_f}$ & 20 & 100 & 200 \\ \hline
$\rho_s[10^3 \frac{kg}{m^3}]$ & 1 & 10 & 1 \\ \hline
$\nu_s$ & 0.4 & 0.4 & 0.4 \\ \hline
$\mu_s[10^6 \frac{m^2}{s}]$ & 0.5 & 0.5 & 2 \\ \hline
\end{tabular}
\end{table}

\subsubsection*{FSI1}
\begin{table}[H]
\centering
\caption{Results of FSI 1 test case}
\label{my-label}
\begin{tabular}{|l|l|l|l|l|l|}
\hline
Cells & Dofs & $d_x(A) [\times10^{-3}]$ & $d_y(A)[\times10^{-3}]$ & Drag & Lift \\ \hline
2474 & 21749 & 0.0229 & 0.8265 & 14.0581 & 0.7546 \\ \hline
7307 & 63365 & 0.02309 & 0.7797 & 14.1077 & 0.7518 \\ \hline
11556 & 99810 & 0.02295 & 0.8249 & 14.2046 & 0.7613 \\ \hline
\textbf{ref} & \textbf{ref} & \textbf{0.0227} & \textbf{0.8209} & \textbf{14.295} & \textbf{0.7638} \\ \hline
\end{tabular}
\end{table}

\subsubsection*{FSI2}

\begin{table}[H]
\centering
\caption{FSI2 test case results, $\Delta t = 0.01$, using the harmonic lifting operator}
\label{FSI2_table}
\begin{tabular}{|l|l|l|l|l|l|}
\hline
Cells & Dofs & $d_x(A) [\times10^{-3}]$ & $d_y(A) [\times10^{-3}]$ & Drag & Lift \\ \hline
2474 & 21749 & $-15.26 \pm 13.44$ & $1.34 \pm 82.38$ & $157.02 \pm 14.79 $ & $-1.426 \pm 258.4 $ \\ \hline
7307 & 63365 & $-14.96 \pm 13.24$ & $1.01 \pm 81.67$ & $159.01 \pm 16.33$ & $1.88 \pm 254.2 $ \\ \hline
11556 & 99810 & $-14.96 \pm 13.23 $ & $1.29 \pm 81.9 $ & $ 161.09 \pm 17.66 $ & $0.06 \pm 255.78 $ \\ \hline
\textbf{ref} & \textbf{ref} & $\textbf{-14.58} \pm \textbf{12.44}$ & $\textbf{1.23} \pm \textbf{80.6}$ & $\textbf{208.83} \pm \textbf{73.75}  $ & $\textbf{0.88} \pm \textbf{234.2} $ \\ \hline
\end{tabular}
\end{table}

\begin{table}[H]
\centering
\caption{FSI-2 with $\Delta t = 0.001$, using harmonic lifting operator}
\label{FSI2_table_0001}
\begin{tabular}{|l|l|l|l|l|l|}
\hline
Cells & Dofs & $d_x(A) [x10^{-3}]$ & $d_y(A) [x10^{-3}]$ & Drag & Lift \\ \hline
2474 & 21749 & $ -15.10 \pm 13.32 $ & $1.16 \pm 82.46 $ & $ 159.53 \pm 17.44 $ & $ 0.68 \pm 259.10 $ \\ \hline
7307 & 63365 & $ -14.85 \pm 13.14 $ & $1.21 \pm 81.72 $ & $ 160.72 \pm 17.84  $ & $0.93 \pm 255.14 $ \\ \hline
11556 & 99810 & $ -14.83  \pm 13.11  $ & $ 1.24 \pm 81.6 $ & $ 161.50 \pm 18.17  $ & $0.62 \pm 254.40  $ \\ \hline
\textbf{ref} & \textbf{ref} & $\textbf{-14.58} \pm \textbf{12.44}$ & $\textbf{1.23} \pm \textbf{80.6}$ & $\textbf{208.83} \pm \textbf{73.75}  $ & $\textbf{0.88} \pm \textbf{234.2} $ \\ \hline
\end{tabular}
\end{table}

The tables \ref{FSI2_table} and \ref{FSI2_table_0001} shows results for the FSI2 test case. The displacements in both results for both of the $\Delta t =0.01$ and $\Delta t = 0.001$ show convergence towards the \textbf{ref} values. The lift converges more slowly towards the ref value, while the drag values are off by about a value of 45 in both cases.

\begin{figure}[H]
\includegraphics[scale=0.35,trim={0mm 0mm 0mm 0mm},clip]{./Verification_Validation/Hron_Turek/FSI2_d_970.png}
\caption{Deformation at $t =9.70 $sec. The bar is marked with pink colour and deformed using warp by vector in paraview.}
\end{figure}
\begin{figure}[H]
\includegraphics[scale=0.35,trim={0mm 0mm 0mm 0mm},clip]{./Verification_Validation/Hron_Turek/FSI2_u_970.png}
\caption{Fluid velocity at $ t = 9.70 $sec on the reference mesh}
\end{figure}

\begin{figure}[H] 
  \label{FSI2_plots} 
  \begin{minipage}[b]{0.6\linewidth}
    \centering
    \includegraphics[width=0.9\linewidth]{./Verification_Validation/Hron_Turek/FSI2_dis_x.png} 
    \caption{Displacement x} 
    \vspace{4ex}
  \end{minipage}%%
  \begin{minipage}[b]{0.6\linewidth}
    \centering
    \includegraphics[width=0.9\linewidth]{./Verification_Validation/Hron_Turek/FSI2_dis_y.png} 
    \caption{Displacement x} 
    \vspace{4ex}
  \end{minipage} 
  \begin{minipage}[b]{0.6\linewidth}
    \centering
    \includegraphics[width=0.9\linewidth]{./Verification_Validation/Hron_Turek/FSI2_drag.png} 
    \caption{Drag} 
    \vspace{4ex}
  \end{minipage}%% 
  \begin{minipage}[b]{0.6\linewidth}
    \centering
    \includegraphics[width=0.9\linewidth]{./Verification_Validation/Hron_Turek/FSI2_lift.png} 
    \caption{Lift} 
    \vspace{4ex}
  \end{minipage} 
  \caption{Plots of FSI2 result values for, $\Delta t = 0.001$, with 11556 elements}
\end{figure}

 and figure \ref{}

\begin{comment}
\begin{table}[H]
\centering
\caption{FSI-2 with $\Delta t = 0.01$}
\label{my-label}
\begin{tabular}{|l|l|l|l|l|l|l|l|l|}
\hline
Cells & Dofs & $d_x(A) [\times 10^{-3}]$ & $d_y(A) [\times10^{-3}]$ & Drag & Lift & Extrapolation & BC & $\Delta t$ \\ \hline
2698 & 23563 & $-15.17 \pm 13.35  $ & $1.13 \pm 82.5$ & $160.12 \pm 17.88$ & $0.87 \pm 259.62$ & Biharmonic & 1 & 0.01 \\ \hline
2698 & 23563 & $-14.92 \pm 13.17 $ & $1.13 \pm 81.86$ & $160.28 \pm 17.94$ & $0.65 \pm 254.07$ & Biharmonic & 2 & 0.01 \\ \hline
2698 & 23563 & $-15.10 \pm 13.32 $ & $1.16 \pm 82.46 $ & $159.53 \pm,17.44 $ & $ 0.68 \pm 259.10 $ & Harmonic & - & 0.01 \\ \hline
10792 & 92992 & $-14.83 \pm 13.11 $ & $1.24 \pm 81.72$ & $ 161.50 \pm 18.17  $ & $0.89 \pm 255.10 $ & Harmonic & - & 0.01 \\ \hline
\textbf{ref} & \textbf{ref} & $\textbf{-14.58} \pm \textbf{12.44}$ & $\textbf{1.23} \pm \textbf{80.6}$ & $\textbf{208.83} \pm \textbf{73.75}  $ & $\textbf{0.88} \pm \textbf{234.2} $ & \textbf{ref} & \textbf{ref} & \textbf{ref} \\ \hline
\end{tabular}
\end{table}


\begin{table}[H]
\centering
\caption{FSI-2 with $\Delta t = 0.001$}
\label{my-label}
\begin{tabular}{|l|l|l|l|l|l|l|}
\hline
Cells & Dofs & $d_x(A) [\times10^{-3}]$ & $d_y(A) [\times10^{-3}]$ & Drag & Lift  \\ \hline
2698 & 23563 & $ 15.42 \pm 13.10$ & $1.14 \pm 83.39$ & $157.01 \pm 15.69$ & $ -0.77 \pm 274.36$  \\ \hline
10792 & 92992 & $ 15.16 \pm 12.94$ & $ 1.20 \pm 82.5 $ & $ 158.26 \pm 16.03$ & $ -0.09 \pm 267.81$  \\ \hline
\textbf{ref} & \textbf{ref} & $\textbf{-14.58} \pm \textbf{12.44}$ & $\textbf{1.23} \pm \textbf{80.6}$ & $\textbf{208.83} \pm \textbf{73.75}  $ & $\textbf{0.88} \pm \textbf{234.2} $ \\ \hline
\end{tabular}
\end{table}

\begin{figure}[H]
\includegraphics[scale=0.40,trim={0mm 0mm 0mm 0mm},clip]{./Verification_Validation/Hron_Turek/FSI2_d_920.png}
\caption{Deformation at $t =9.20 $sec. The bar is marked with pink colour and deformed using warp by vector in paraview.}
\end{figure}
\begin{figure}[H]
\includegraphics[scale=0.40,trim={0mm 0mm 0mm 0mm},clip]{./Verification_Validation/Hron_Turek/FSI2_u_920.png}
\caption{Fluid velocity at $ t = 9.20 $sec on the reference mesh}
\end{figure}

\subsubsection*{FSI3}
\begin{table}[H]
\centering
\caption{FSI3 with $\Delta t = 0.01$}
\label{my-label}
\begin{tabular}{|l|l|l|l|l|l|}
\hline
 &  & $d_x(A) [\times10^{-3}]$ & \& $d_y(A)[\times10^{-3}]$ & Lift & Drag \\ \hline
0 & bi\_bc2 & $-1.74 \pm 1.766$ & $ 3.565e \pm 26.01 $ & $ -1.357 \pm 138.74$ & $439.41 \pm12.218$ \\ \hline
 & bi\_bc1 & $-1.77 \pm 1.793$ & $ 3.582 \pm 26.21 $ & $ -1.843 \pm 139.29$ & $439.60\pm12.218$ \\ \hline
 & laplace & $-1.791 \pm 1.807$ & $ 3.290 \pm 26.48 $ & $1.960 \pm 142.306$ & $ 439.35 \pm 12.04$ \\ \hline
 &  &  &  &  &  \\ \hline
1 & bi\_bc2 & $-2.397 \pm 2.406$ & $ 1.774 \pm 32.282 $ & $ 3.668 \pm 150.02$ & $ 449.71 \pm 18.172$ \\ \hline
 & bi\_bc1 & $ -2.440 \pm 2.446$ & $ 1.766 \pm 32.562 $ & $ 3.929 \pm 150.76$ & $ 449.98 \pm 18.030$ \\ \hline
 & laplace & $ -2.481 \pm 2.486$ & $ 1.655 \pm 32.851$ & $ 3.318 \pm 153.58$ & $ 449.74 \pm18.052$ \\ \hline
ref. &  & $ ?2.69 \pm 2.53$ & $1.48 \pm 34.38 $ & $ 2.22 \pm 149.78$ & $ 457.3 \pm 22.66$ \\ \hline
\end{tabular}
\end{table}
\end{comment}


\subsubsection*{FSI3 results}
\begin{table}[H]
\centering
\caption{FSI3 unsteady test case results with $\Delta t = 0.01$, with harmonic lifting operator}
\label{my-label}
\begin{tabular}{|l|l|l|l|l|l|}
\hline
Cells & Dofs & $d_x (A)[\times10?3 ]$ & $d_y (A)[\times10?3 ]$ & Drag & Lift \\ \hline
2474 & 21249 & $-1.79 \pm 1.80$ & $3.29 \pm 2.64$ & $439.36 \pm 12.04$ & $1.96 \pm 142.31$ \\ \hline
7307 & 63365 & $-2.48 \pm 2.48$ & $ 1.64 \pm 3.28$ & $449.77 \pm 18.02$ & $3.41 \pm 153.47$ \\ \hline
11556 & 99810 & $ -2.47 \pm 2.45$ & $ 1.27 \pm 3.28$ & $456.60 \pm 18.73$ & $1.55 \pm 153.46$ \\ \hline
ref & ref & $\bold{-2.69 \pm  2.56}$ & $\bold{1.48 \pm 34.38}$ & $\bold{457.3 \pm 22.66}$ & $\bold{2.22 \pm 149.78}$ \\ \hline
\end{tabular}
\end{table}

\subsection*{Comment on FSI tests}
 A very important thing to notice about this paper \cite{Hron2006a} is that it is a proposal for a benchmark, as it is called \say{Proposal for numerical benchmarking of fluid-structure interaction between an elastic object and laminar incompressible flow}. So the point of the paper is give a specific problem setup which others can contribute result-data. A paper was published in 2010 by J. Hron, Turek, et al, \cite{Turek2010} that compared results of different discretizations and solution approaches. This paper \cite{Turek2010} gives 7 different methods and results for two of the FSI test cases. They state in the numerical results that \say{However, also clear differences between the different approaches with regard to accuracy are visible. Particularly for the drag and lift values, which lead to differences of up to order 50\%, and also for the displacement valueswhich are in the range of 10\% errors.}. With this in mind it is important to know that the referential values used are only those reported from the original paper, which only looked at one implementation. While the paper which compares results, only 2 of the 7 contributions were schemes of monolithic nature, which are the closest one should refer to in this thessis.  In the Appendix \ref{sec:bigboys} is a copy of the results from the paper comparing schemes, showing different results for different discretization, with different time steps and unknowns. 


The FSI1 test gives a low fluid velocity and gives very low displacements. FSI1 is therefore not a rigid test for FSI. In fact I experienced in the beginning of making the FSI solver, that even with a wrong implementation i got good FSI1 results. However it is a good test for early checks, because if FSI1 is wrong the rest will definitely not work.  \newline

For the FSI2 test case we only have results from the initial Hron and Turek paper \cite{Hron2006a}. As previously stated the results for the FSI3 case differ by in some cases 50\% for Drag and Lift. With this in mind, in the FSI2 case I am off by less then 10\% for displacement in x and y direction and for Lift. While Drag is off by about  50\%. It is reasonable to assume that since there were such differences in the results for different implementations for the FSI3 results, we would expect similar behavior in the FSI2 results.

If we compare the results reported in the FSI3 case in the inital paper by Hron and Turek 2006, to their reported results in 2010 \ref{sec:bigboys} implementation 3, we can see that they do not report the same results same scheme, leading one believe that that they have changed their implementation a bit. 

A note should be added about the construction of the computational meshes. If we look at figures \ref{fig:fullmesh} and \ref{fig:partmesh}. The node spacing on the inlet is small to ensure that the parabolic inlet profile is upheld. There is also smaller node spacing around the circle and around the bar, leading to larger node spacing as we move down stream. In the unsteady CFD and FSI cases there is vortex shedding happening to the right of the bar. With large node spacing in this area the vortexes may not be produced to its full extent, hence introducing errors in the unsteady results. 

In hindsight, larger gaps in the number of elements between each mesh should have been larger. The three meshes that are mainly used go from 2474 cells to 7307 cells to 11556 cells. When calculating in 2D to see converging effects in the results of smaller node spacing, one should make meshes with 4 times the number of cells for each new mesh. This might have helped in converging to the referential values.

\begin{comment}
FSI2 has a Reynolds number of 100 with medium fluid speeds. It does however induce great deformations and is therefore a great FSI test. Using a mesh motion technique that upholds the mesh structure is crucial. The results shown are with the Harmonic extension but I got similar results for biharmonic extension. The FSI2 test could be run with a fairly large time $\Delta t = 0.01 $, since the fluid velocity is not very high. And it can been seen from the results that choosing a time step $\Delta t = 0.001$ did not yield much better results. The Drag was computed to be about a value of 40 less than reported by Hron and Turek. This i believe is because a combination of the way the mesh was constructed, without a good enough boundary layer, and the mapping to a reference configuration. +++ \newline

Lastly the FSI3 test has a Reynolds number of 200 with greater inflow speeds. The run time improvements such as reusing the jacobian did not work with the FSI3 test. The Jacobian needed to be more precise than for FSI2. For this reason a different mesh, with lower number of cells needed to be used in FSI3.  +++
\end{comment}

















%\newpage
\newpage
\section{Flexible tube}
This benchmark details fluids in flexible cylindrical tubes \cite{Greenshields2005}, which is relevant to practical problems as pressure surge in pipelines and blood flow in arteries. Specifically this deals with wave propagation due to pressure drop. Pressure will be added to one side of the cylinder producing a wave in which theoretical wave and flow speed can be calculated and compared to numerical findings.

\subsection{Problem definition}
\begin{center} 
\includegraphics[scale=0.6,trim={0 15cm 0 10},clip]{./Verification_Validation/Flexible_tube/Pictures/definition.pdf}
\end{center}

The properties and geometry were selected for its representation of blood flow in a large artery. 

\subsection{Boundary conditions}


\subsection{Quanities for comparison}

\subsection{Results}
\begin{table}[h!]
\centering
\caption{My caption}
\label{my-label}
\begin{tabular}{|l|l|l|l|l|l|l|l|l|l|l|}
\hline
l {[}mm{]} & D {[}mm{]} & t {[}mm{]} & E {[}MPa{]} & $K_s [kPa] $ & $K_f [GPa]$ & $\nu_s $ & $\rho_s [\frac{kg}{m^3}]$ & $ \mu_s [kPa]  $ & $\rho_f []$ & $\mu_f [\frac{Ns}{m^2}]$ \\ \hline
100 & 20 & 2 & 1 & 833 & 2.2 & 0.3 & 1000 & 385 & 1000 & 0.0004 \\ \hline
\end{tabular}
\end{table}

\subsection{Mesh motion techniques}\label{sec:mesh_motion}
In this section we compare different mesh motion techniques from \ref{sec:meshmotion}. The test will be run using a version of the CSM test discussed earlier. The tests will compare the different techniques by looking at the how the deformation is lifted into the fluid domain. This is done by looking at a plot of the mesh after deformation to see how much cells distort and why. This is done using Paraview.\newline
In these test cases we have the fluid initially at rest and with no inflow on the fluid. A gravitational force is applied to the structure much like the previous CSM test. The only difference is that we now use the full domain from the \ref{sec:HronTurek} . The tests are run as time-dependent with a the backward Euler scheme, leading to a steady state solution. In the first test case the parameters from CSM1 are used, and in the CSM4 the gravitational force has just been increased from 2 to 4. 
\subsection*{Boundary conditions}
The upper, lower and left boundary is set as no slip, that is no velocity in the fluid. On the left boundary there is a do nothing, and zero pressure. 
\subsection*{Quantities for comparison}
The different techniques will be plotted with the minimal value of the Jacobian. The Jacobian is if we remember the determinant of the deformation rate. If the jacobian is zero anywhere in the domain it means that the cells overlap and can cause singularity in the matrices during assembly. \newline
We will also look at the a plot of the deformation in the domain. To visualize the how the different mesh motion techniques work. It is possible to see that if get thin triangles in the computational domain then the mesh motion operator is no good. 

\subsubsection*{CSM1}
\begin{figure}[H]
\label{fig:fluid_structure}
\caption{CSM1}
\includegraphics[scale=0.60, trim={0mm 0mm 0mm 0mm},clip]{./Verification_Validation/Mesh_motion_results/CSM1.png}
\end{figure}

\begin{figure}[H]  \label{fig:CSM1_pictures} 
  \caption {CSM1 with different techniques}
  \begin{minipage}[b]{0.5\linewidth}
    \centering
    \includegraphics[scale=0.2]{./Verification_Validation/Mesh_motion_results/CSM1_laplace.png} 
    \caption{Harmonic smart} 
    \vspace{4ex}
  \end{minipage}%%
  \begin{minipage}[b]{0.5\linewidth}
    \centering
    \includegraphics[scale=0.2]{./Verification_Validation/Mesh_motion_results/CSM1_constant.png} 
    \caption{Harmonic constant} 
    \vspace{4ex}
  \end{minipage} 
  \begin{minipage}[b]{0.5\linewidth}
    \centering
    \includegraphics[scale=0.2]{./Verification_Validation/Mesh_motion_results/CSM1_bibc1.png} 
    \caption{Biharmonic bc1} 
    \vspace{4ex}
  \end{minipage}%% 
  \begin{minipage}[b]{0.5\linewidth}
    \centering
    \includegraphics[scale=0.2]{./Verification_Validation/Mesh_motion_results/CSM1_bibc2.png} 
    \caption{Biharmonic bc2} 
    \vspace{4ex}
  \end{minipage} 
\end{figure}

\begin{table}[H]
\centering
\caption{Displacement of CSM1 test}
\label{my-label}
\begin{tabular}{|l|l|l|}
\hline
Technique & $d_y(A) [\times 10^{-3}]$ & $d_x(A) [\times 10^{-3}]$ \\ \hline
Harmonic & 65.406 & 7.036 \\ \hline
Constant & 43.033 & 2.999 \\ \hline
Bibc1 & 65.404 & 7.036 \\ \hline
Bibc2 & 65.405 & 7.036 \\ \hline
\end{tabular}
\end{table}





\section{Investigating Numerical stability for Fluid-Structure Interaction Problems}
The following section will give a brief insight in to the effects of choosing different $\theta$ values in the $\theta-$scheme for different time steps. 
The benchmark tests FSI2 and FSI3, as discussed in the previous section, has been investigated since they are known to be numerical unstable for certain values of $\theta$ and $\Delta t$. Only the effects of Drag as been studied as the three other quantities shows similar behavior. 
The impact of different $\theta$ values on energy stability in the solid mechanical benchmark CSM3 is also investigated.\newline

\ref{fig:FSI2drag_plots} show the plots of Drag with $\Delta t = 0.01$, showing the instability when choosing $\theta = 0.5$. The Crank-Nicholson scheme is stable until about 13 seconds where we see that it is numerically unstable and the solver diverges. While the shifted Crank-Nicholson, $\theta = 0.5 + \Delta t$, is stable throughout the computing time.\newline

\begin{figure}[H] 
  \begin{minipage}[b]{0.6\linewidth}
    \centering
    \includegraphics[scale=0.40]{./Temporal_stability/FSI2_001_051_big.png} 
    \caption{$\theta = 0.50 + \Delta t $} 
    \vspace{4ex}
  \end{minipage}%%
  \begin{minipage}[b]{0.6\linewidth}
    \centering
    \includegraphics[scale=0.40]{./Temporal_stability/FSI2_001_050_big.png} 
    \caption{$\theta = 0.50 $} 
    \vspace{4ex}
  \end{minipage} 
  \begin{minipage}[b]{0.6\linewidth}
    \centering
    \includegraphics[scale=0.40]{./Temporal_stability/FSI2_001_051_small.png} 
    \caption{$\theta = 0.50 +\Delta t $} 
    \vspace{4ex}
  \end{minipage}%% 
  \begin{minipage}[b]{0.6\linewidth}
    \centering
    \includegraphics[scale=0.40]{./Temporal_stability/FSI2_001_050_small.png} 
    \caption{$\theta = 0.50 $} 
    \vspace{4ex}
  \end{minipage} 
\caption {Drag for FSI2 with $\Delta t = 0.01$ with different values for $\theta$}
\label{fig:FSI2drag_plots} 
\end{figure}

Figures \ref{fig: FSI3_long_short} show drag for FSI3 simulation with $\Delta t = 0.001$ and $\theta = 0.5$, showing long term stability for the normal Crank-Nicholson scheme.

\begin{figure}[H]
\begin{tabular}{ll}
\includegraphics[scale=0.4]{./Temporal_stability/FSI3_0001_050_big.png}
&
\includegraphics[scale=0.4]{./Temporal_stability/FSI3_0001_050_small.png}
\end{tabular}
\caption{FSI3 drag plot, with $\Delta t = 0.001$ and $\theta =0.5$ showing long term numerical stability}
\label{fig: FSI3_long_short}
\end{figure}

For the CSM3 case only the solid bar is computed, and with an applied force g and no friction, the bar should move down and bounce back up infinitely, for a correct solution.

Figure \ref{fig:CSM3_dis_plots} shows plots of the displacements in x and y directions for $\theta = 0.5$ and $1$. With the implicit scheme ($\theta=1$) the bar moves to a steady state solution. This means energy has not been preserved and the energy dissipates. While in the Crank-Nicholson scheme ($\theta = 0.5$), the bar moves down and back up. This indicates that the Crank-Nicholson scheme is energy preserving.

\begin{figure}[H] 
  \begin{minipage}[b]{0.6\linewidth}
    \centering
    \includegraphics[scale=0.40]{./Temporal_stability/CSM3_implicit.png} 
    \caption{$\theta = 1 $} 
    \vspace{4ex}
  \end{minipage}%%
  \begin{minipage}[b]{0.6\linewidth}
    \centering
    \includegraphics[scale=0.40]{./Temporal_stability/CSM3_implicit_y.png} 
    \caption{$\theta = 1 $} 
    \vspace{4ex}
  \end{minipage} 
  \begin{minipage}[b]{0.6\linewidth}
    \centering
    \includegraphics[scale=0.40]{./Temporal_stability/CSM3_Crank.png} 
    \caption{$\theta = 0.5 $} 
    \vspace{4ex}
  \end{minipage}%% 
  \begin{minipage}[b]{0.6\linewidth}
    \centering
    \includegraphics[scale=0.40]{./Temporal_stability/CSM3_Crank_y.png} 
    \caption{$\theta = 0.5 $} 
    \vspace{4ex}
  \end{minipage} 
 \label{fig:CSM3_dis_plots} 
 \caption {CSM3 displacements with $\Delta t = 0.01$ with different values for $\theta$}
\end{figure}


\subsubsection*{Discussion on numerical stability}
The shifted version of the Crank-Nicholson scheme is stable when computing for time step values as low as $\Delta t = 0.01$. However with $\Delta t = 0.001$ the normal Crank-Nicholson scheme ($\theta =0.5$) can be used and is long term stable. 
It has also been reported by Wick 2011 \cite{Wick2011} that the Crank-Nicholson, $\theta = 0.5$, scheme is stable throughout the computing time by setting $\Delta t = 0.001$.

In the FSI2 case the results for the finest mesh showed, in previous chapter, similar results for $\Delta t = 0.01$ and $\Delta t = 0.001$, meaning that the shifted version of the Crank-Nicholson scheme can be applied, in certain cases, with $\Delta t = 0.01$ greatly reducing computational runtime.

The CSM3 test shows that choosing $\theta = 0.5$ is crucial for preserving energy when computing solid problems.















