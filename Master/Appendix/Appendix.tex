\chapter{Appendix}
\section{Lagrangian description of solid mechanics}

\begin{center}
\includegraphics[scale=0.4]{continuum_mapping.png}
\end{center}
Let $ \hat{\mathcal{S}}$, $\mathcal{S}$, $\mathcal{S}(t)$ be the initial stress free configuration of a given body, the reference and the current configuration respectively.
I define a smooth mapping from the reference configuration to the current configuration:
\begin{equation}
\chi^s(\textbf{X},t) : \hat{\mathcal{S}} \rightarrow \mathcal{S}(t)     
\end{equation}
Where $\textbf{X}$ denotes a material point in the reference domain and $\chi^s$ denotes the mapping from the reference configuration to the current configuration. Let $d^s(\textbf{X},t)$ denote the displacement field which describes deformation on a body. The mapping $\chi^s$ can then be specified from a current position plus the displacement from that position:
\begin{equation} \label{eq:chi}
 \chi^s(\textbf{X},t) = \textbf{X}  + d^s(\textbf{X} ,t) 
\end{equation}
which can be written in terms of the displacement field:
\begin{equation}
 d^s(\textbf{X},t) = \chi^s(\textbf{X},t) -\textbf{X}   
\end{equation}

Let w(\textbf{X},t) be the domain velocity which is the partial time derivative of the displacement: 
\begin{equation}
 w(\textbf{X},t) = \frac{\partial \chi^s(\textbf{X},t)}{\partial t}   
\end{equation}

\subsection{Deformation gradient}
The deformation gradient describes the rate at which a body undergoes deformation.
Let $d(\textbf{X},t)$ be a differentiable deformation field in a given body, the deformation gradient is then:  
\begin{equation}
\label{eq:deformation_gradient}
F = \frac{\partial \chi^s(\textbf{X},t)}{\partial \textbf{X}} = \frac{\partial \textbf{X}  + d^s(\textbf{X} ,t) }{\partial \textbf{X}} =  I + \nabla d(\textbf{X},t) 
\end{equation}
which denotes relative change of position under deformation in a Lagrangian frame of reference. We can observe that when there is no deformation. The deformation gradient $F$ is simply the identity matrix. \newline

Let the Jacobian determinant, which is the determinant of the of the deformation gradient F, be defined as:
\begin{equation}\label{eq:J}
J = \text{det}(F)
\end{equation}
The Jacobian determinant is used to change between volumes, assuming infinitesimal line and area elements in the current $ds, dx$ and reference $dV,dX$ configurations. The Jacobian determinant is therefore known as a volume ratio.

\begin{comment}
The volume elements $dv, dV$ can be expressed by the dot product:
\begin{equation}
 dv = ds\cdot dx = J dS dX
\end{equation}
This is used to get the Nansons formula:
\begin{equation}\label{eq:Nanson}
ds = JF^{-T}dS
\end{equation}
which holds for an arbitrary line element in different configurations. This will be useful later on when describing equations in different configurations.
\end{comment}

\subsection{Strain}
\begin{figure}[H]
\includegraphics[scale=0.40]{./Solid_equations/Strain.png}
\caption{Deformation of a line element with length $d\epsilon$ into a line element with length $\lambda d \epsilon$}
\end{figure}

Strain is the relative change of location between two particles. Strain, strain rate and deformation is used to describe the relative motion of particles in a continuum. This is the fundamental quality that causes stress \cite{Richter2010}.

Observing two neighboring points \textbf{X} and \textbf{Y}. Let \textbf{Y} be described by adding and subtracting the point \textbf{X} and rewriting \textbf{Y} from the point \textbf{X} plus a distance $d\textbf{X}$   :
\begin{equation}
\textbf{Y} = \textbf{Y} + \textbf{X} - \textbf{X} = \textbf{X} + |\textbf{Y} - \textbf{X}| \frac{\textbf{Y} - \textbf{X}}{|\textbf{Y} - \textbf{X}|} = \textbf{X} + d\textbf{X}
\end{equation}

Let $d\textbf{X}$ be denoted by:
\begin{align}
d\textbf{X} =& d\epsilon \textbf{a}_0\\
d\epsilon =& |\textbf{Y} - \textbf{X}| \\
\textbf{a}_0 =& \frac{\textbf{Y} - \textbf{X}}{|\textbf{Y} - \textbf{X}|}
\end{align}
where $d\epsilon$ is the distance between the two points and $\textbf{a}_0$ is a unit vector 

We see now that $d\textbf{X}$ is the distance between the two points times the unit vector. \newline

A certain motion transform the points $\textbf{Y}$ and $\textbf{X}$ into the displaced positions $\textbf{x} = \chi^s(\textbf{X},t)$ and $\textbf{y} = \chi^s(\textbf{Y},t)$. Using Taylor?s expansion \textbf{y} can be expressed in terms of the deformation gradient:

\begin{align}
\textbf{y} =& \chi^s(\textbf{Y},t) = \chi^s(\textbf{X} + d\epsilon \textbf{a}_0,t) \\
=& \chi^s(\textbf{X},t) + d\epsilon F \textbf{a}_0 + \mathcal{O}(\textbf{Y}-\textbf{X}) 
\end{align}

where $\mathcal{O} (\textbf{Y}-\textbf{X})$ refers to the small error that tends to zero faster than $(\textbf{X} - \textbf{Y}) \rightarrow \mathcal{O}$. \newline

Setting $\textbf{x} = \chi^s(\textbf{X},t)$  It follows that:
\begin{align}
\textbf{y} - \textbf{x} =&  d\epsilon F \textbf{a}_0 + \mathcal{O}(\textbf{Y}-\textbf{X}) \\
=& F(\textbf{Y} - \textbf{X}) + \mathcal{O}(\textbf{Y}-\textbf{X}) 
\end{align}

Let the $\textbf{stretch vector}$ be $\lambda_{\textbf{a}0}$, which goes in the direction of $\textbf{a}_0$: 
\begin{equation}
\lambda_{\textbf{a}0}(\textbf{X},t) = F(\textbf{X},t)\textbf{a}_0 
\end{equation}

Looking at the square of $\lambda$:
\begin{align}
\lambda^2 &=  \lambda_{\textbf{a}0} \lambda_{\textbf{a}0} = F(\textbf{X},t)\textbf{a}_0 F(\textbf{X},t)\textbf{a}_0 \\
&= \textbf{a}_0 F^TF\textbf{a}_0 = \textbf{a}_0 C \textbf{a}_0
\end{align}

Introducing the right Cauchy-Green tensor:
 \begin{equation}
 C = F^TF
\end{equation}
Since $\textbf{a}_0 $ is just a unit vector, we see that C measures the squared length of change under deformation. We see that in order to determine the stretch one needs only the direction of $\textbf{a}_0$ and the tensor C.
C is also symmetric and positive definite $C = C^T$.  I also introduce the Green-Lagrangian strain tensor E:
\begin{equation}\label{eq:StrainTensor}
E = \frac{1}{2}(C - I) 
\end{equation}
which is also symmetric since C and I are symmetric. The Green-Lagrangian strain tensor E has the advantage of having no contributions when there is no deformations. Where the Cauchy-Green tensor gives the identity matrix for zero deformation.
		
\subsection{Stress}
While strain, deformation and strain rate only describe the relative motion of particles in a given volume, stress give us the internal forces between neighboring particles. Stress is responsible for deformation and is therefore crucial in continuum mechanics. The unit of stress is force per area.

Introducing the Cauchy stress tensor $\sigma_s$, which define the state of stress inside a material. The version of Cauchy stress tensor is defined by the material model used. 
If we use this tensor on an area, taking the stress tensor times the normal vector $\sigma_s \bold{n}$ we get the forces acting on that area.

Stress tensor defined from the Cauchy by the constitutive law of St. Venant-Kirchhoff hyperelastic material model: 
\begin{equation}
 \sigma_s = \frac{1}{J} F(\lambda_s (tr E)I + 2\mu_sE) F^T
\end{equation}

Using the deformation gradient and the Jacobian determinant., I get the first Piola-Kirchhoff stress tensor P:
\begin{equation}
 P = J \sigma F^{-T} 
\end{equation}
This is known as the \textit{Piola Transformation} and maps the tensor into a Lagrangian formulation which will be used when stating the solid equation.

Introducing the second Piola-Kirchhoff stress tensor S:
\begin{equation}
S = J F^{-1}\sigma F^{-T} = F^{-1} P = S^T 
\end{equation}
from this relation the first Piola-Kirchhoff tensor can be expressed by the second:
\begin{equation}
P = FS
\end{equation}


\chapter{Results}\label{sec:bigboys}
\begin{figure}[H]
\includegraphics[scale=0.7,trim={12cm 0cm 16.3cm 5.5cm},clip]{./Appendix/BigBoyResults.png}
\caption{Results from different contributions in from the paper Turek et.al 2010 \cite{Turek2010}}
\end{figure}


\lstdefinelanguage{Python}{
 keywords={typeof, null, catch, switch, in, int, str, float, self},
 ndkeywords={boolean, throw, import},
 ndkeywords={return, class, if ,elif, endif, while, do, else, True, False , catch, def},
 ndkeywordstyle=\color{blue}\bfseries,
 identifierstyle=\color{black},
 sensitive=false,
 comment=[l]{\#},
 morecomment=[s]{/*}{*/},
 commentstyle=\color{purple}\ttfamily,
 stringstyle=\color{red}\ttfamily,
 backgroundcolor = \color{lightgray}
}

\chapter{FSI Implementation in FEniCS}
Here we will look at the implementation of the monolithic FSI Code in FENiCS. Not every part of the code will be handled here, but hopefully enough so that someone familiar with FEniCS could in short time implement the scheme.

\section{Mesh, mappings and stress tensors}
The mesh was made with Gmesh, with a clear straight boundary splitting the fluid and solid domains. This is converted to a xml file and loaded into FEniCS. 
\begin{lstlisting}[language=Python, basicstyle=\small]
mesh_file = Mesh("Mesh/fluid_new.xml")
\end{lstlisting}

The boundaries are defined using built in domain functions. We only here show the inlet where we specified where the spatial points are, and give it a mark value to be used when the Dirichlet conditions are set. This is also done using CellFunctions to define the fluid and structure domain. 

\begin{lstlisting}[language=Python,basicstyle=\small]
Inlet = AutoSubDomain(lambda x: "on_boundary" and near(x[0],0))
boundaries = FacetFunction("size_t",mesh_file)
Inlet.mark(boundaries, 3)
\end{lstlisting}


All the mappings are made with python functions, this is done so we can call the mappings later in the variational form.

\begin{lstlisting}[language=Python, basicstyle=\small]
def F_(d):
	return (Identity(len(d)) + grad(d))

def J_(d):
	return det(F_(d))
\end{lstlisting}

The stress tensors are also defined as functions to be used in the variational form.

\begin{lstlisting}[language=Python, basicstyle=\small]
def E(d):
	return 0.5*(F_(d).T*F_(d) - Identity(len(d)))

def S(d,lamda_s,mu_s):
    I = Identity(len(d))
    return 2*mu_s*E(d) + lamda_s*tr(E(d))*I

def Piola1(d,lamda_s,mu_s):
	return F_(d)*S(d,lamda_s,mu_s)
	
def sigma_f_u(u,d,mu_f):
    return  mu_f*(grad(u)*inv(F_(d)) + inv(F_(d)).T*grad(u).T)

def sigma_f_p(p, u):
    return -p*Identity(len(u))
    
\end{lstlisting}

\section{Variational form}

The variational form can be written directly into FEniCS. We write all the forms and add them together to make one big form to be calculated in the upcoming timeloop. We start by looking at the fluid variational form
\begin{lstlisting}[language=Python, basicstyle=\small]
J_theta = theta*J_(d_["n"]) + (1 - theta)*J_(d_["n-1"])

F_fluid_linear = rho_f/k*inner(J_theta*(v_["n"] - v_["n-1"]), psi)*dx_f

F_fluid_nonlinear =  Constant(theta)*rho_f*\
inner(J_(d_["n"])*grad(v_["n"])*inv(F_(d_["n"]))*v_["n"], psi)*dx_f

F_fluid_nonlinear += inner(J_(d_["n"])*sigma_f_p(p_["n"], d_["n"])*\
inv(F_(d_["n"])).T, grad(psi))*dx_f

F_fluid_nonlinear += Constant(theta)*inner(J_(d_["n"])\
*sigma_f_u(v_["n"], d_["n"], mu_f)*inv(F_(d_["n"])).T, grad(psi))*dx_f

F_fluid_nonlinear += Constant(1 - theta)*inner(J_(d_["n-1"])*\
sigma_f_u(v_["n-1"], d_["n-1"], mu_f)*inv(F_(d_["n-1"])).T, grad(psi))*dx_f

F_fluid_nonlinear += \
inner(div(J_(d_["n"])*inv(F_(d_["n"]))*v_["n"]), gamma)*dx_f

F_fluid_nonlinear += Constant(1 - theta)*rho_f*\
inner(J_(d_["n-1"])*grad(v_["n-1"])*inv(F_(d_["n-1"]))*v_["n-1"], psi)*dx_f

F_fluid_nonlinear -= rho_f*inner(J_(d_["n"])*\
grad(v_["n"])*inv(F_(d_["n"]))*((d_["n"]-d_["n-1"])/k), psi)*dx_f
\end{lstlisting}
where $dx_f$ is the fluid domain. The theta value choses the scheme we want. inv() gives the inverse of the matrix. 


Next is the solid variational form:
\begin{lstlisting}[language=Python, basicstyle=\small]
delta = 1E10
F_solid_linear = rho_s/k*inner(v_["n"] - v_["n-1"], psi)*dx_s +\
delta*(1/k)*inner(d_["n"] - d_["n-1"], phi)*dx_s -\
delta*inner(Constant(theta)*v_["n"] + Constant(1-theta)*v_["n-1"], phi)*dx_s

F_solid_nonlinear = inner(Piola1(Constant(theta)*d_["n"] +\
Constant(1 - theta)*d_["n-1"], lamda_s, mu_s), grad(psi))*dx_s
\end{lstlisting}

The weighed delta is talked about in section \ref{Discretization}

\section{NewtonSolver}
To solve a non-linear problem we need make a newton solver, taken from [Mikael kompendium]. F is derivated wrt to $dvp$ and is assembled to a matrix. -F is assembled as b and we solve until the residual is smaller than a give tolerance. There is also an if test which only assembles the Jacobian the first and tenth time. This reuses the Jacobian to improve speed, as we shall see later. Lastly the the mpi line is when the code is running in parallell that we only print out the values for one of the computational nodes.  
\begin{lstlisting}[language=Python, basicstyle=\small]
    Iter      = 0
    residual   = 1
    rel_res    = residual
    chi = TrialFunction(DVP)
    Jac = derivative(F, dvp_["n"], chi)

    while rel_res > rtol and residual > atol and Iter < max_it:
        if Iter == 0 or Iter == 10:
            A = assemble(Jac, tensor=A)#, keep_diagonal = True)
            A.ident_zeros()
            
        b = assemble(-F)
        
        [bc.apply(A, b, dvp_["n"].vector()) for bc in bcs]
        solve(A, dvp_res.vector(), b)
                   

        dvp_["n"].vector()[:] = dvp_["n"].vector()[:] + lmbda*dvp_res.vector()[:]
        [bc.apply(dvp_["n"].vector()) for bc in bcs]
        
        rel_res = norm(dvp_res, 'l2')
        residual = b.norm('l2')

        if MPI.rank(mpi_comm_world()) == 0:
            print "Newton iteration %d: r (atol) = %.3e (tol = %.3e), r (rel) = %.3e (tol = %.3e) " \
        % (Iter, residual, atol, rel_res, rtol)
        Iter += 1
\end{lstlisting}


\section{Timeloop}
In the time loop we call on the solver and update the functions $ v,d,p$ for each round. The counter value is used when we only want to take out values every certain number of time iterations.

\begin{lstlisting}[language=Python, basicstyle=\small]

while t <= T:
    print "Time t = %.5f" % t
    time_list.append(t)
    if t < 2:
        inlet.t = t;
    if t >= 2:
        inlet.t = 2;

    #Reset counters
    atol = 1e-6;rtol = 1e-6; max_it = 100; lmbda = 1.0;

    dvp = Newton_manual(F, dvp, bcs, atol, rtol, max_it, lmbda,dvp_res,VVQ)


    times = ["n-2", "n-1", "n"]
    for i, t_tmp in enumerate(times[:-1]):
   	dvp_[t_tmp].vector().zero()
    	dvp_[t_tmp].vector().axpy(1, dvp_[times[i+1]].vector())

    t += dt
    counter +=1
\end{lstlisting}


	






