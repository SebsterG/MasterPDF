\chapter{Continuum mechanics in different frames of reference}
All matter is made up of small building blocks called atoms, and the mathematics of the fundamental laws at this tiny scale is called quantum mechanics. The characteristic scales at which quantum mechanics operate is in the nanometer range, which is far smaller than most length scales found in nature.
If we zoom out to the \textit{macroscale}, the scale on which phenomena are large enough to be visible for the human eye, matter like solids and fluids behave statistically uniform, and it is reasonable to assume the matter can be mathematically described, and modeled, as a continuum. The continuum hypothesis states that there exist no space inside the matter and the matter completely fills up the space it occupies. 
We can use mathematics with basic Newtonian physical laws to model fluids and solids when they are assumed continuous. These newtonian laws are generally expressed in two frames of reference, Lagrangian and Eulerian, depending on the the scientific branch. \newline

To exemplify these frameworks we can imagine a river running down a mountain.	
In the Eulerian framework we are the observer standing besides the river looking at the flow. We are not interested in each fluid particle but only how the fluid acts as a whole flowing down the river. The Eulerian description of fluid mechanics is advantageous as we can imagine the fluid continuously deforming along the river side. \newline

The Lagrangian description is particularly beneficial for describing solid mechanical problems, as one is generally.

In the Lagrangian description we have to imagine ourselves on a leaf going down the river with the flow.  The Lagrangian description is particularly beneficial for describing solid mechanical problems, as one is generally interested in where the solid particles are in relation to each other. A solid mechanical example is modeling a deflecting beam attached to a wall. To model deflection one needs to know where all the particles are compared to each other. The more force that is applied the more the particles move in relation to each other. The Lagrangian description helps us to easier model a problem where need to know the particles spatial-relation to each other \newline

This chapter introduces briefly the fluid and solid equations with their respective boundary conditions. A detailed description of the Lagrangian framework and the stress and strain relations are covered in the appendix.

\section{Conservation of mass and momentum for solid matter}
With matter assumed continuous, fundamental physical laws like conservation of mass and conservation of momentum can be applied to derive a differential equation describing the motions of a solid. Information about the particular material of a solid is added through constitutive relations.
The differential solid equation will be stated in the Lagrangian reference system \cite{Holzapfel2000}, in the solid domain $\mathcal{S}$ as:
\begin{equation}\label{eq:Solid}
\rho_s \frac{\partial \bold{d}^2}{\partial t^2} = \nabla \cdot ( P ) + \rho_s f  \hspace{4mm} in \hspace{2mm} \mathcal{S}
\end{equation}
written in terms of the deformation $\bold{d}$, of the solid.
P is the first Piola-Kirchhoff stress tensor, its derivation is included in the appendix A1 for the sake of completeness.
Body forces are denoted as $f$, and are forces that originate outside the body and act on the mass of the body e.g. gravitational force. $\rho_s$ is the solid density, and $\frac{\partial}{\partial t^2}$ is the second time derivative. 

\begin{comment}
\subsection*{Locking}
The problem og shear locking can happen FEM computations with certain elements. 
[mek4250 Kent] - Locking occurs if  $ \lambda >> \nu $ that is, the material is nearly incompressible. The reason is that all the elements discussed in this course are poor at approximating the divergence. Locking refers to the case where the displacement is to small because the divergence term essentially lock the displacement. It is a numerical artifact not a physical feature. [Verbatum]
\end{comment}

\section{Fluid equations}
The Navier-Stokes equations are derived using principles of mass and momentum conservation. These equations describes the velocity and pressure in a given fluid continuum. They are here written in the time domain $\mathcal{F}$:
\begin{align}
\label{eq:NS}
\rho\frac{\partial u}{\partial t} + \rho u \cdot \nabla u &= \nabla \cdot \sigma_f + f \\
\nabla \cdot u &= 0
\end{align}
where $u$ is the fluid velocity, $p$ is the fluid pressure$, \rho$ stands for constant density, f is body force and $ \sigma_f = \mu_f (\nabla u + \nabla u^T)  - pI$ \\
We will only compute incompressible fluids. \\
There does not yet exist an analytical solutions to the N-S equations, only simplified problems can be solved \cite{White2000}. But this does not stop us from discretizing and solving them numerically. \\
Before these equations can be solved we need to impose boundary conditions.
\subsection*{Boundary conditions}
On the Dirichlet boundary $ \partial \mathcal{F}_D$ we impose a given value. This can be initial conditions or set to zero as on walls with "no slip" condition. These conditions needs to be defined for both $u$ and $p$
$$  u = u_0 \text{   on   } \partial \mathcal{F}_D  $$
$$  p = p_0 \text{   on   } \partial \mathcal{F}_D  $$
The forces on the boundaries need to equal an eventual external force $ \bold{f}$
$$ \sigma \cdot \bold{n} = f \text{   on   } \partial \mathcal{F}_N    $$








\section{Fluid and Structure Boundary conditions}
To complete the fluid and solid equations, boundary conditions need to be imposed. The fluid flows and the solid moves within the boundaries noted as $ \partial\mathcal{F}$ and $ \partial \mathcal{S}$ respectively. 

On the Dirichlet boundary, $ \partial \mathcal{F}_D$ and $ \partial \mathcal{S}_D$, we impose a given value. Dirichlet conditions can be initial conditions or set values, such as zero on the fluid boundary for a "no slip" condition. The Dirichlet conditions are defined for $\bold{u}$ and $p$ :
\begin{align}
\bold{u} =& u_0 \text{   on   } \partial \mathcal{F}_D  \\
p =& p_0 \text{   on   } \partial \mathcal{F}_D  \\
\bold{d} =& d_0 \text{ on   } \partial \mathcal{S}_D  \\
\bold{w}(\textbf{X},t)_0 =& \frac{\partial \bold{d}(t=0)}{\partial t} \text{   on   } \partial \mathcal{S}_D   
\end{align}
The forces on the boundaries equal a possible external force $ \bold{f}$. These are are enforced on the Neumann boundaries $\partial \mathcal{F}_N$ and  $\partial \mathcal{S}_N$ :
\begin{align}
\sigma \cdot \bold{n} &= f \text{   on   } \partial \mathcal{F}_N \\   
P\cdot \bold{n} &= f \text{   on   } \partial \mathcal{S}_N    
\end{align}



\begin{comment}
Let $\Omega \in \mathbb{R}^d $ for $d \in \{1,2\}$, be a bounded domain with boundary $ \partial \Omega$. The domain is made up of of two sub domains $ \mathcal{F} $ for the fluid domain, and $\mathcal{S}$ for the solid. The interface between the domains are denoted by $ \Sigma = \mathcal{F} \cap \mathcal{S} $. The reference or initial is denoted by $ \hat{\Sigma} = \hat{\mathcal{F}} \cap \hat{\mathcal{S}}  $ 
\end{comment}

