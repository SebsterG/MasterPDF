\chapter{Continuum mechanics in different frames of reference}
All materials are made up of atoms, and between atoms there is space. The laws that govern these atoms are complex and are very difficult to model. However materials like solids and fluids can be modeled if we assume them to exist as a continuum. This means that we assume that there exist no space inside the materials and they fill completely up the space they occupy. 
We can use mathematics with basic physical laws to model fluids and solids when they are assumed continuous. These laws are generally expressed in two frames of reference, Lagrangian and Eulerian. To exemplify these frameworks we can imagine a river running down a mountain. In the Eulerian framework we are the observer standing still besides the river looking at the flow. We are not interested in each fluid particle but only how the fluid acts as a whole flowing down the river. This approach fits the fluid problem as we can imagine the fluid continuously deforming along the river side. \newline

In the Lagrangian description we have to imagine ourselves on a leaf going down the river with the flow. Looking out as the mountain moves and we stand still compared to the fluid particles. This description fits a solid problem nicely since we are generally interested in where the solid particles are in relation to each other. For instance modeling a beam attached to a wall at one end and a weight at the other end. We can imagine the beam bending and to model this bending we need to where all the particles are compared to each other. The more force that is applied the more the particles move in relation to eachother. \newline

In this chapter I will introduce both of these frameworks and the equations which are needed to model Fluid and Structure separately. I start with the Lagrangian description, by providing a short introduction to Lagrangian physics for the sake of completeness. Then introduce the solid and fluid equations. For a more detailed look on Lagrangian physics and the solid equation see \cite{Holzapfel2000}.


\chapter{Solid Equations}
In this chapter we will look briefly into the solid equation. The solid equation is most commonly described in a Lagrangian description. In a Lagrangian description the material particles are fixed with the grid points, this is a useful property when tracking the solid domain.

\section*{Lagrangian physics}
\subsection*{Mapping and identites}
We will start by providing a short introduction to Lagrangian physics for the sake of completeness.
\begin{center}
\includegraphics[scale=0.4]{continuum_mapping.png}
\end{center}
We define $ \hat{\mathcal{S}}$ as the initial stress free configuration of a given body, $\mathcal{S}$ as the reference and $\mathcal{S}(t)$ as the current configuration respectively.
We need to define a smooth mapping from the reference configuration to the current configuration:
\begin{equation}
\chi^s(t) : \hat{\mathcal{S}} \rightarrow \mathcal{S}(t)     
\end{equation}
Following the notation of \cite{Holzapfel2000},where $\textbf{X}$ denote a material point in the reference domain and $\chi^s$ denotes the mapping from the reference configuration. 
Let $d^s(\textbf{X},t)$ denote the displacement field and w(\textbf{X},t) the domain velocity. We then set the mapping
\begin{equation}
 \chi^s(\textbf{X},t) = \textbf{X}  + d^s(\textbf{X} ,t) 
\end{equation}

where $d^s(\textbf{X} ,t)$ represents displacement field
\begin{equation}
 d^s(\textbf{X},t) = \chi^s(\textbf{X},t) -\textbf{X}   
\end{equation}
and the domain velocity is the partial time derivative: 
\begin{equation}
 w(\textbf{X},t) = \frac{\partial \chi^s(\textbf{X},t)}{\partial t}   
\end{equation}

Next we will need a function that describes the rate of deformation in the solid.
\subsection*{Deformation gradient}
When a continuum body undergoes deformation and is moved from the reference configuration to some current configuration we need a deformation gradient that describes the rate of deformation in the body. 
If $d(\textbf{X},t)$ is a differentiable deformation field in a given body. We define the deformation gradient as:  
\begin{equation}
\label{eq:deformation_gradient}
F = \frac{\partial \chi}{\partial \textbf{X}} = I + \nabla d(\textbf{X},t) 
\end{equation}
which denotes relative change of position under deformation in a Lagrangian frame of reference.  

A change in volume between reference ($dv$) and current ($dV$) configuration is defined as :
\begin{align}
\label{eq:det_F}
dv =& J dV\\
J =& \text{det}(F)
\end{align}
Where J is the determinant of the deformation gradient F known as the Jacobian determinant or volume ratio. If there is no motion, that is $ F = I$ and $J=1$ , there is no change in volume. But we can also have the constraint $J=1$ with motion, but preserving the volume.
If we assume infinitesimal line and area elements in the current $ds, dx$ and reference $dV,dX$ configurations. The volume elements $dv, dV$ can be expressed by the dot product:
\begin{equation}
 dv = ds\cdot dx = J dS dX
\end{equation}
This is used to get the Nansons formula:
\begin{equation}\label{eq:Nanson}
ds = JF^{-T}dS
\end{equation}
which holds for an arbitrary line element in different configurations.

\subsection*{Strain}
In continuum mechanics relative change of location of particles is called strain and this is the fundamental quality that causes stress in a material. \cite{Richter2016}. An import strain measure is the right Cauchy-Green tensor 
$$C = F^TF$$ 
which is symmetric and positive definite $C = C^T$.  We also introduce the Green-Lagrangian strain tensor E:
$$E = \frac{1}{2}(F^TF -I) $$
which is also symmetric since C and I are symmetric. This measures the squared length change under deformation.
		
\subsection*{Stress}
Stress is the internal forces between neighboring particles. We introduce the Cauchy stress tensor:
\begin{equation}
 \sigma_s = \frac{1}{J} F(\lambda_s (tr E)I + 2\mu_sE) F^T
\end{equation}
Using \eqref{eq:Nanson} we get the first Piola-Kirchhoff stress tensor P:
\begin{equation}
 P = J \sigma F^{-T} 
\end{equation}
We also introduce the second Piola-Kirchhoff stress tensor S:
\begin{equation}
S = J F^{-1}\sigma F^{-T} = F^{-1} P = S^T 
\end{equation}
from this relation we can write the first Piola-Kirchhoff tensor by the second:
\begin{equation}
P = FS
\end{equation}
\subsection*{Solid equation}
From the principles of conservation of mass and momentum, we get the solid equation stated in the Lagrangian reference system (Following the notation from \cite{Richter2016}):
\begin{equation}\label{eq:Solid}
\rho_s \frac{\partial d^2}{\partial t^2} = \nabla \cdot ( P ) + \rho_s f 
\end{equation}
where we used the first Piola-Kirchhoff stress tensor.

\subsection*{Locking}
The problem og shear locking can happen FEM computations with certain elements. 
[mek4250 Kent] - Locking occurs if  $ \lambda >> \nu $ that is, the material is nearly incompressible. The reason is that all the elements discussed in this course are poor at approximating the divergence. Locking refers to the case where the displacement is to small because the divergence term essentially lock the displacement. It is a numerical artifact not a physical feature. [Verbatum]








\section{Fluid equations}
The fluid equation will be stated in an Eulerian framework. In this framework the domain has fixed points where the fluid passes through. 
The Navier-Stokes(N-S) equations are, like the solid equation, derived using principles of mass and momentum conservation. N-S describes the velocity and pressure in a given fluid continuum. They are here written in the fluid time domain $\mathcal{F}$ as an incompressible fluid:
\begin{align}
\label{eq:NS}
\rho_f\big( \frac{\partial u}{\partial t} +  u \cdot \nabla u\big) &= \nabla \cdot \sigma_f + f  \\
\nabla \cdot u &= 0
\end{align}
where $u$ is the fluid velocity, $p$ is the fluid pressure$, \rho$ stands for density which, will be kept constant. f is body force and $\sigma_f$ is the Cauchy stress tensor, $ \sigma_f = \mu_f (\nabla u + \nabla u^T)  - pI$, where $I$ is the Identity matrix. \\

There does not yet exist an analytical solutions to the N-S equations, only simplified problems can be solved \cite{White2000}. Actually there is a prize set out by the Clay Mathematics Institute of 1 million dollars to whomever can show the existence and smoothness of Navier-Stokes solutions \cite{Fefferman2000}, as apart of their millennium problems. 
Nonetheless this does not stop us from discretizing and solving N-S numerically. The field of modeling fluids numerically is known as Computational Fluid Dynamics(CFD). And CFD is extensively used in for instance weather forecasts, construction of aircraft, and biomedical engineering.\\

One difficulty in the N-S equations is the nonlinearity appearing in the convection term on the left hand side. Non-linearity is most often handled using Newtons method or Picard iteration.

Before these equations can be solved we need to impose boundary conditions.
\subsection{Fluid Boundary conditions}
Lastly I need to impose boundary conditions. The fluid flows within the boundary noted as $ \partial \mathcal{F}$. On the Dirichlet boundary $ \partial \mathcal{F}_D$ we impose a given value. This can be initial conditions or set to zero as on walls with "no slip" condition. These conditions are defined for $u$ $p$ and $d$:
$$  u = u_0 \text{   on   } \partial \mathcal{F}_D  $$
$$  p = p_0 \text{   on   } \partial \mathcal{F}_D  $$

The forces on the boundaries need to equal an eventual external force $ \bold{f}$. These are are enforced on the Neumann boundaries:
$$ \sigma \cdot \bold{n} = f \text{   on   } \partial \mathcal{F}_N    $$









\begin{comment}
Let $\Omega \in \mathbb{R}^d $ for $d \in \{1,2\}$, be a bounded domain with boundary $ \partial \Omega$. The domain is made up of of two sub domains $ \mathcal{F} $ for the fluid domain, and $\mathcal{S}$ for the solid. The interface between the domains are denoted by $ \Sigma = \mathcal{F} \cap \mathcal{S} $. The reference or initial is denoted by $ \hat{\Sigma} = \hat{\mathcal{F}} \cap \hat{\mathcal{S}}  $ 
\end{comment}

