\section{Fluid and Structure Boundary conditions}
In order to obtain a unique solution to the fluid and solid equations, we need to specify a computational domain and impose boundary conditions. The fluid flow and the solid moves within the boundaries noted as $ \partial\mathcal{F}$ and $ \partial \mathcal{S}$, respectively. 

Dirichlet boundary conditions, often referred to as essential ones, are defined on the boundaries $ \partial \mathcal{F}_D$ and $ \partial \mathcal{S}_D$.
Dirichlet boundary conditions can be fixed or time varying values, such as zero at the fluid boundary for a "no slip" condition, or a Womersley profile \cite{He1997} at the inlet of a pipe. For a problem to be well posed we need also to prescribe initial conditions.
The Dirichlet boundary conditions are defined for $\bold{u}$ and $p$ as :
\begin{align}
\bold{u} =& u_0 \text{   on   } \partial \mathcal{F}_D  \\
p =& p_0 \text{   on   } \partial \mathcal{F}_D  \\
\bold{d} =& d_0 \text{ on   } \partial \mathcal{S}_D  \\
\bold{w}(\textbf{X},t)_0 =& \frac{\partial \bold{d}(t=0)}{\partial t} \text{   on   } \partial \mathcal{S}_D   
\end{align}

In addition, there are Neumann boundary conditions, often referred to as natural, which states a specific value of the derivative of a solution at the boundary. More specifically, $\partial \mathcal{F}_N$ and $\partial \mathcal{S}_N$. One can also control eventual forces on the Neumann boundary, to possibly equal an external force $ \bold{f}$:
\begin{align}
\sigma \cdot \bold{n} &= f \text{   on   } \partial \mathcal{F}_N \\   
P\cdot \bold{n} &= f \text{   on   } \partial \mathcal{S}_N    
\end{align}

For the sake of completeness, it should be noticed that there exists other boundary conditions as well, for instance the Robin boundary conditions, which are the Dirichlet and Neumann conditions combined.
