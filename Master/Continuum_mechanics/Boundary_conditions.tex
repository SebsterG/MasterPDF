\section{Fluid and Structure Boundary conditions}
To complete the fluid and solid equations, boundary conditions need to be imposed. The fluid flows and the solid moves within the boundaries noted as $ \partial\mathcal{F}$ and $ \partial \mathcal{S}$ respectively. 

On the Dirichlet boundary, $ \partial \mathcal{F}_D$ and $ \partial \mathcal{S}_D$, we impose a given value. Dirichlet conditions can be initial conditions or set values, such as zero on the fluid boundary for a "no slip" condition. The Dirichlet conditions are defined for $\bold{u}$ and $p$ :
\begin{align}
\bold{u} =& u_0 \text{   on   } \partial \mathcal{F}_D  \\
p =& p_0 \text{   on   } \partial \mathcal{F}_D  \\
\bold{d} =& d_0 \text{ on   } \partial \mathcal{S}_D  \\
\bold{w}(\textbf{X},t)_0 =& \frac{\partial \bold{d}(t=0)}{\partial t} \text{   on   } \partial \mathcal{S}_D   
\end{align}
The forces on the boundaries equal a possible external force $ \bold{f}$. These are are enforced on the Neumann boundaries $\partial \mathcal{F}_N$ and  $\partial \mathcal{S}_N$ :
\begin{align}
\sigma \cdot \bold{n} &= f \text{   on   } \partial \mathcal{F}_N \\   
P\cdot \bold{n} &= f \text{   on   } \partial \mathcal{S}_N    
\end{align}
