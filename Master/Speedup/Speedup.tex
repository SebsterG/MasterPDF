\chapter{Compute time reduction techniques}\label{runtime}
Computational fluid and solid mechanics are both well know to be time consuming or computationally intensive individually, and the iterative monolithic FSI problem is naturally an order of magnitude more expensive. As demonstrated in chapter \ref{chap:VV}, both accuracy and high resolutions are necessary to obtain adequate results. However, it was quickly observed that simulations took days to complete, even on meshes that were coarser than reported in this thesis. It was evident that simulations with adequate mesh and time step size would result in infeasible compute times. 

The initial implementation was therefore profiled, and the Newton solver of the monolithic FSI problem was identified to spend the most compute time. The point of this chapter is not to rigorously experiment with multiple speed-up techniques, but merely to describe implementations of the compute time reduction techniques that made simulation times feasible. Retrospectively it was discovered that the same techniques were applied previously\cite{Sciences2012}, but there only briefly addressed. 

\section{Newton runtime profile}
Table \ref{no_opt} shows compute times for the FSI1 problem for the initial FEniCS implementation. The table contains the amount of time spent on assembly of the Jacobian, assembly of the residual, and a call to solve the linear system of equations. It can be observed that almost 95\% of the resources are spent of assembly of the Jacobian, and efforts to reduce this bottleneck was taken. The next sections introduce two ways of speeding up assembly; the first is reuse of the Jacobian and the second is reduction of quadrature degree. The computational speedup will be compared to Table \ref{no_opt}.

\begin{table}[H]
\centering
\caption{Newton solver timed with no optimizations, $\bold{\Delta t = 0.5}$ }
\label{no_opt}
\begin{tabular}{|l|l|l|l|}
\hline
Method               & \textbf{Runtime {{[}}s{{]}}} & Runtime {{[}}\%{{]}} & \textbf{Calls} \\ \hline
Assembly of Jacobian & \textbf{60.7}                    & 94.4                     & \textbf{5}     \\ \hline
Assembly or residual & \textbf{0.6}                     & 0.9                      & \textbf{5}     \\ \hline
Solve                & \textbf{2.8}                     & 4.4                      & \textbf{5}     \\ \hline
Fulltime             & \textbf{64.3}                    & 100\%                    & -                \\ \hline
\end{tabular}
\end{table}

\section{Reusing the Jacobian}
As the time step size is relatively low in all the benchmark experiments, the effects of reassembling the Jacobian at every iteration was hypothesized to have a negligible impact on convergence. It was therefore tested whether a reuse of the Jacobian could speed up computations, by assembling only a few specified times. Table \ref{jac_reuse} shows the same simulation as presented in table \ref{no_opt}, but now  using $\Delta t = 0.5$. Even with a time step size that is fairly large, we obtain an acceptable decrease in compute time by -74\%. The effect was an increase in the number of iterations, but and overall decreased compute time since the Jacobian was assembled only once. 

\begin{table}[H]
\centering
\caption{Parts of the Newtonsolver timed with reuse of the jacobian run with $\bold{\Delta t = 0.5}$}
\label{jac_reuse}
\begin{tabular}{|l|l|l|l|}
\hline
Method & \textbf{Runtime {[}s{]}} & Runtime {[}\%{]} & \textbf{Calls} \\ \hline
Assembly of Jacobian & \textbf{11.5 (-80\%)} & 68.7 & \textbf{1 (-20\%)} \\ \hline
Assembly or residual & \textbf{0.9 (+50\%)} & 5.8 & \textbf{9 (+46\%)} \\ \hline
Solve & \textbf{4.2 (+50\%)} & 25.0 & \textbf{9 (+46\%)} \\ \hline
Fulltime & \textbf{16.8 (-74 \%)} & - & - \\ \hline
\end{tabular}
\end{table}


\section{Quadrature reduction}
Assembly of the Jacobian matrix with non-linear terms induces a high number of quadrature points, dense matrix, and lower convergence rates. However, faster convergence rates can be obtained if the accuracy of the Jacobian is reduced by specifying the of quadrature degree. Table \ref{tab:quadreduce} shows that reducing the quadrature degree gives a 92 \% decrease in time spent assembling the Jacobian even with the same number of calls. The full time spent went down by 87 \%. The effect is therefore more iterations per time step, but an overall decrease in compute time as both convergence of the linear solvers and assembly of the residual, is much faster, respectively. However, it should be noted that reduction of the quadrature degree can lead to numerical divergence of the system for some problems.\newline

\begin{table}[H]
\centering
\caption{Parts of the Newtonsolver timed with quadrature reduxe run with FSI1 ,$\bold{\Delta t = 0.5}$}
\label{tab:quadreduce}
\begin{tabular}{|l|l|l|l|}
\hline
Method & \textbf{Runtime {[}s{]}} & Runtime {[}\%{]} & \textbf{Calls} \\ \hline
Assembly of Jacobian & \textbf{4.9 (-92\%)} & 60.3 & \textbf{5 (0\%)} \\ \hline
Assembly or residual & \textbf{0.5 (-17\%)} & 6.9 & \textbf{5 (0\%)} \\ \hline
Solve & \textbf{2.6 (-7\%)} & 31.9 & \textbf{5 (0\%)} \\ \hline
Fulltime & \textbf{8.2 (-87\%)} & - & - \\ \hline
\end{tabular}
\end{table}


\section{Summary of runtime improvement techniques}
Finally, a combination of Jacobian reuse and reduction of the quadrature is presented in Table \ref{both_tech}, where the total compute time decreased by 89\%. Both techniques were applied to the FSI2 and FSI3 problems as well.

\begin{table}[H]
\centering
\caption{Newton solver timed with jacobian reuse and quadrature reduce run with $\bold{\Delta t = 0.5}$}
\label{both_tech}
\begin{tabular}{|l|l|l|l|}
\hline
Method & \textbf{Runtime {[}s{]}} & Runtime {[}\%{]} & \textbf{Calls} \\ \hline
Assembly of Jacobian & \textbf{1.2 (-98\%)} & 18.1 & \textbf{1 (-20\%)} \\ \hline
Assembly or residual & \textbf{0.9 (+50\%)} & 14.7 & \textbf{9 (+46\%)} \\ \hline
Solve & \textbf{4.4 (+57\%)} & 66.2 & \textbf{9 (+46\%)} \\ \hline
Fulltime & \textbf{6.6 (-89\%)} & - & - \\ \hline
\end{tabular}
\end{table}















