\chapter{Conclusions}
The work on this thesis started with a plan to build a FSI solver meant to be applied to a hemodynamical problem. The plan was to build a monolithic solver to obtain reference data, and a partitioned scheme, specifically the scheme introduced by Fernandez 2013 \cite{Fernandez2013}. It was discovered how difficult it is to build an accurate monolithic FSI solver that can handle large deformations and high fluid velocity. This complexity can be seen in the variability of the results provided in the appendix \ref{sec:bigboys}, from highly renowned scientists. An added difficulty is that the field of FSI is still very young, meaning good information is not yet accessible. Work on the partitioned scheme was put aside and full focus was set on building, verifying, and validating a monolithic solver in a rigorous manner. Lessons were learned about the importance of lifting operators, energy stable numerical schemes and the need to compute on reference domains.

For the FSI fields future the main problem of long run time should be tackled by looking at partitioned schemes. When a FSI scheme is partitioned we can use legacy code for the structure and fluid parts, increasing run times significantly. The difficulty is still to transfer the energies between the fluid and structure when each problem is solved sequentially. Explicit coupling schemes are known to be unconditionally unstable for standard Dirichlet-Neumann strategies when there is a large amount of added-mass in the system \cite{Fernandez2015, VanBrummelen2009}. There are however, schemes which offer added-mass free stability with explicit coupling, where the interface is treated through a Robin-Neumann coupling. The first scheme was made for coupling of a thin walled structure by Fernandez et al., 2013 \cite{Fernandez2013}. Later a scheme was made with an extension coupling with a thick walled structure by Fernandez et.al,2015 \cite{Fernandez2015}. The scheme for coupling thick walled structures is rather complex. Hopefully my work on a building a monolithic solver can help others advance the field of FSI. 
