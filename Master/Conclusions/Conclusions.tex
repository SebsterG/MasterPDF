\chapter{Conclusions}
The intended work in this thesis was admittedly to develop a fast partitioned computational FSI framework to investigate possible high-frequent arterial wall vibrations in arteries in the vicinity of the brain, stemming from flow instabilities \cite{valen2011direct,valen2014mind}. The initial plan was to implement and use a monolithic FSI solver, but solely to obtain reference data. The idea was to implement partitioned scheme, specifically the scheme introduced by Fernandez 2013 \cite{Fernandez2013}. However, it was soon discovered that there are many difficulties in implementing an accurate monolithic FSI solver that can handle large deformations and high fluid velocity. This complexity is reflected by the variability of the results provided in the appendix \ref{sec:bigboys}, from highly renowned scientists. In addition, the scientific field of computational FSI is still in its infancy, and appropriate reference data is inaccessible. The work started out broadly, addressing monolithic and partitioned schemes simultaneously. However, work on the partitioned scheme was ultimately put aside and the focus during the last six months was on rigorous implementation, verification, and validation of a monolithic solver. Lessons were learned about the importance of lifting operators, energy stable numerical schemes and the need to compute on reference domains.

For the scientific field of computational FSI, it seems the most critical challenge to overcome is stability of partitioned schemes. The partitioned approach allows for used of legacy code for the structure and fluid parts, decreasing computing times significantly. The difficulty is still to transfer the stresses between the fluid and structure when each problem is solved sequentially. Explicit coupling schemes are known to be unconditionally unstable for standard Dirichlet-Neumann strategies when there is a large amount of added-mass in the system \cite{Fernandez2015, VanBrummelen2009}. There are however, schemes that offer added-mass free stability with explicit coupling, where the interface is treated through a Robin-Neumann coupling. The first scheme was developed for coupling of a thin walled structure\cite{Fernandez2013}. Later a scheme was developed with an extension coupling with a thick walled structure, by the same main author \cite{Fernandez2015}. The partitioned scheme for coupling thick walled structures is rather complex, but with the use of the associated code attached to this thesis, a monolithic solver can hopefully help others advance the field of FSI. 
