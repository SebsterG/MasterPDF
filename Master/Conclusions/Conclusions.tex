\chapter{Insights, conclusions, and future work}
\section{Insights}
When I started working on my thesis the plan was to build a Fluid-Structure Interaction solver meant to be applied to a hemodynamic problem. The plan was to first build a monolithic solver to get reference data, to then move on to a partitioned approach, specifically the scheme implemented by Fernandez 2013 \cite{Fernandez2013}. Starting from scratch I thought building the monolithic solver would be easy, so I worked on the monolithic scheme and the partitioned scheme in parallel, this process took a long time. I eventually figured out how difficult it is to build an accurate monolithic FSI solver from scratch that can handle large deformations and high fluid velocity. Eventually only focusing on building the monolithic solver, and validating the solver in a rigorous manner. When starting from scratch I had to make all the mistakes needed to understand what it takes to build a solver that could handle large deformations, understand the importance good lifting operator has on FSI and the need for a long time stable scheme. An added difficulty is that the field of FSI is still very young so there are not many scientist researching the topic, meaning good information is not yet accessible.

\section{Future work}
Computing FSI problems is difficult for a many reasons but the main reason FSI computing is not used in a commercial sense yet is the long run time for simulations. For future work the problem of long run time should be tackled by looking at partitioned schemes. When a FSI scheme is partitioned we can use existing solvers for the structure and fluid parts, increasing run times significantly. The difficulty is still to transfer the energies between the fluid and structure when each problem is solved separately.  

Explicit coupling schemes are known to be unconditionally unstable for standard Dirichlet-Neumann strategies when there is a large amount of added-mass in the system \cite{Fernandez2015, VanBrummelen2009}. There are however, schemes which offer added-mass free stability with explicit coupling, where the interface is treated through a Robin-Neumann coupling. The first scheme was made for coupling of a thin walled structure by Fernandez et al., 2013 \cite{Fernandez2013}. Later a scheme was made with an extension coupling with a thick walled structure by Fernandez et.al,2015 \cite{Fernandez2015}. The scheme for coupling thick walled structures is rather complex and uses a number of techniques that are out of the scope of this thesis. 

\section{Conclusions}
This thesis has developed a framework for solving Fluid-Structure Interaction problem using a monolithic scheme. I have introduced the mathematics and physics that governs fluids and structures, and the conditions needed to model FSI problems. I have discussed the challenges one meets when trying to model and compute FSI problems. I have introduced the $\theta-$scheme with a shifted version of the Crank-Nicholson scheme (\theta = 0.5 + \Delta t), meant to uphold stability for long time FSI problems. The code has been verified in parts using MMS. \newline

Using the proposal benchmark from Hron and Turek 2010, I have validated the FSI solver first in fluid and structure separately and finally for a full FSI problem. I have compared data to contributions made by others imposing similar schemes. I have shown how crucial the choice of lifting operator is for computing FSI problems with moderate to large deformations, and the need for a stable scheme for long run times.

