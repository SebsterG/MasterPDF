\section{Conservation of Mass and Momentum for Fluid}
The differential equations describing the velocity and pressure in a fluid are called the Navier-Stokes (N-S) equations. The N-S equations are derived, like the solid equation, from principles of mass and momentum conservation, assuming fluid to act as a continuum.
The fluid equations are stated in an Eulerian framework. 
The N-S equations are written in the fluid time domain $\mathcal{F}(t)$ as an incompressible fluid:
\begin{align}
\label{eq:NS}
\rho_f\big( \frac{\partial \bold{u}}{\partial t} +  \bold{u} \cdot \nabla \bold{u}\big) &= \nabla \cdot \sigma_f + \rho_ff  \hspace{4mm} in \hspace{2mm} \mathcal{F}(t)\\
\nabla \cdot \bold{u} &= 0 \hspace{4mm} in \hspace{2mm} \mathcal{F}(t)
\end{align}
where $\bold{u}$ is the fluid velocity, $p$ is the fluid pressure$, \rho_f$ stands for density. f is body force and $\sigma_f$ is the Cauchy stress tensor, $ \sigma_f = \mu_f (\nabla \bold{u} + \nabla \bold{u}^T)  - pI$, setting $\mu_f$ to be constant hence denoting a Newtonian fluid. $I$ denotes the identity matrix. \\

There does not yet exist an analytical solutions to the Navier-Stokes equations for every fluid problem. 
Analytical solutions can only be found for fluid problems with certain boundary conditions and geometries using the N-S equations \cite{White2000}. Actually there is a prize set out by the Clay Mathematics Institute of 1 million dollars to whomever can show the existence and smoothness of Navier-Stokes equations \cite{Fefferman2000}, as a part of their millennium problems. 
Nonetheless this does not stop us from discretizing and solving N-S numerically. One difficulty in the Navier-Stokes equations is the nonlinearity appearing in the convection term on the left hand side. This non-linearity can be handled using Newtons method, or Picard iterations. Another difficulty is finding a suitable equation to solve for the pressure field \cite{Charlesworth2003}. As there is no natural pressure update. 