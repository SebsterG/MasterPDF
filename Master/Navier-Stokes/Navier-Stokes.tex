
\section{Fluid equations}
The fluid equation will be stated in an Eulerian framework. In this framework the domain has fixed points where the fluid passes through. 
The Navier-Stokes equations are like the solid equation derived using principles of mass and momentum conservation. These equations describes the velocity and pressure in a given fluid continuum. They are here written in the fluid time domain $\mathcal{F}$ as an incompressible fluid:
\begin{align}
\label{eq:NS}
\rho_f\big( \frac{\partial u}{\partial t} +  u \cdot \nabla u\big) &= \nabla \cdot \sigma_f + f  \\
\nabla \cdot u &= 0
\end{align}
where $u$ is the fluid velocity, $p$ is the fluid pressure$, \rho$ stands for constant density, f is body force and $ \sigma_f = \mu_f (\nabla u + \nabla u^T)  - pI$ \\
There does not yet exist an analytical solutions to the N-S equations, only simplified problems can be solved \cite{White2000}. But this does not stop us from discretizing and solving them numerically. \\
Before these equations can be solved we need to impose boundary conditions.
\section{Boundary conditions}
Lastly I need to impose boundary conditions for both the solid and fluid equation. The fluid flows within the boundary noted as $ \partial \mathcal{F}$ and the solid moves within $ \partial \mathcal{S}$. The place in which these two meet will be covered in the next chapter. On the Dirichlet boundary $ \partial \mathcal{F}_D$ we impose a given value. This can be initial conditions or set to zero as on walls with "no slip" condition. These conditions are defined for $u$ $p$ and $d$:
$$  u = u_0 \text{   on   } \partial \mathcal{F}_D  $$
$$  p = p_0 \text{   on   } \partial \mathcal{F}_D  $$
$$  d = d_0 \text{   on   } \partial \mathcal{S}_D  $$

The forces on the boundaries need to equal an eventual external force $ \bold{f}$. These are are enforced on the Neumann boundaries:
$$ \sigma \cdot \bold{n} = f \text{   on   } \partial \mathcal{F}_N    $$
$$ P\cdot \bold{n} = f \text{   on   } \partial \mathcal{S}_N    $$







