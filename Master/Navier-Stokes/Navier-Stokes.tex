\chapter*{Fluid equations}
The Navier-Stokes equations are derived using principles of mass and momentum conservation. These equations describes the velocity and pressure in a given fluid continuum. They are here written in the time domain $\mathcal{F}$:
\begin{align}
\rho\frac{\partial u}{\partial t} + \rho u \cdot \nabla u &= \nabla \cdot \sigma_f + f \\
\nabla \cdot u &= 0
\end{align}
where $u$ is the fluid velocity, $p$ is the fluid pressure$, \rho$ stands for constant density, f is body force and $ \sigma_f = \mu_f (\nabla u + \nabla u^T)  - pI$ \\
We will only compute incompressible fluids. \\
There does not yet exist an analytical solutions to the N-S equations, only simplified problems can be solved \cite{White2000}. But this does not stop us from discretizing and solving them numerically. \\
Before these equations can be solved we need to impose boundary conditions.
\subsection*{Boundary conditions}
On the Dirichlet boundary $ \partial \mathcal{F}_D$ we impose a given value. This can be initial conditions or set to zero as on walls with "no slip" condition. These conditions needs to be defined for both $u$ and $p$
$$  u = u_0 \text{   on   } \partial \mathcal{F}_D  $$
$$  p = p_0 \text{   on   } \partial \mathcal{F}_D  $$
The forces on the boundaries need to equal an eventual external force $ \bold{f}$
$$ \sigma \cdot \bold{n} = f \text{   on   } \partial \mathcal{F}_N    $$







