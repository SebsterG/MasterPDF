
\section{Fluid equations}
The fluid equation is stated in an Eulerian framework. In the Eulerian framework the domain has fixed points where the fluid passes through. 
The Navier-Stokes(N-S) equations are, like the solid equation, derived using principles of mass and momentum conservation. N-S describes the velocity and pressure in a given fluid continuum. Written in the fluid time domain $\mathcal{F}$ as an incompressible fluid:
\begin{align}
\label{eq:NS}
\rho_f\big( \frac{\partial u}{\partial t} +  u \cdot \nabla u\big) &= \nabla \cdot \sigma_f + \rho_ff  \hspace{4mm} in \hspace{2mm} \mathcal{F}\\
\nabla \cdot u &= 0 \hspace{4mm} in \hspace{2mm} \mathcal{F}
\end{align}
where $u$ is the fluid velocity, $p$ is the fluid pressure$, \rho$ stands for density which, will be kept constant. f is body force and $\sigma_f$ is the Cauchy stress tensor, $ \sigma_f = \mu_f (\nabla u + \nabla u^T)  - pI$ denoting a newtonian fluid. $I$ is the Identity matrix. \\

There does not yet exist an analytical solutions to the N-S equations for every fluid problem. Only simplified fluid problems can be solved \cite{White2000} analytically using the N-S equations. Actually there is a prize set out by the Clay Mathematics Institute of 1 million dollars to whomever can show the existence and smoothness of Navier-Stokes solutions \cite{Fefferman2000}, as apart of their millennium problems. 
Nonetheless this does not stop us from discretizing and solving N-S numerically. \\

One difficulty in the N-S equations is the nonlinearity appearing in the convection term on the left hand side. Non-linearity is most often handled using a lagging velocity function, Newtons method, or Picard iterations. Another difficulty is finding a suitable equation to solve for the pressure field \cite{Charlesworth}. The difficulty with pressure has been tackled using for instance the incremental pressure correction scheme (IPCS).