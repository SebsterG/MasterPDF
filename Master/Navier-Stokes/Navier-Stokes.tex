
\section{Fluid equations}
The fluid equation will be stated in an Eulerian framework. In this framework the domain has fixed points where the fluid passes through. 
The Navier-Stokes(N-S) equations are, like the solid equation, derived using principles of mass and momentum conservation. N-S describes the velocity and pressure in a given fluid continuum. They are here written in the fluid time domain $\mathcal{F}$ as an incompressible fluid:
\begin{align}
\label{eq:NS}
\rho_f\big( \frac{\partial u}{\partial t} +  u \cdot \nabla u\big) &= \nabla \cdot \sigma_f + f  \\
\nabla \cdot u &= 0
\end{align}
where $u$ is the fluid velocity, $p$ is the fluid pressure$, \rho$ stands for density which, will be kept constant. f is body force and $\sigma_f$ is the Cauchy stress tensor, $ \sigma_f = \mu_f (\nabla u + \nabla u^T)  - pI$, where $I$ is the Identity matrix. \\

There does not yet exist an analytical solutions to the N-S equations, only simplified problems can be solved \cite{White2000}. Actually there is a prize set out by the Clay Mathematics Institute of 1 million dollars to whomever can show the existence and smoothness of Navier-Stokes solutions \cite{Fefferman2000}, as apart of their millennium problems. 
Nonetheless this does not stop us from discretizing and solving N-S numerically. The field of modeling fluids numerically is known as Computational Fluid Dynamics(CFD). And CFD is extensively used in for instance weather forecasts, construction of aircraft, and biomedical engineering.\\

One difficulty in the N-S equations is the nonlinearity appearing in the convection term on the left hand side. Non-linearity is most often handled using Newtons method or Picard iteration.

Before these equations can be solved we need to impose boundary conditions.
\subsection{Fluid Boundary conditions}
Lastly I need to impose boundary conditions. The fluid flows within the boundary noted as $ \partial \mathcal{F}$. On the Dirichlet boundary $ \partial \mathcal{F}_D$ we impose a given value. This can be initial conditions or set to zero as on walls with "no slip" condition. These conditions are defined for $u$ $p$ and $d$:
$$  u = u_0 \text{   on   } \partial \mathcal{F}_D  $$
$$  p = p_0 \text{   on   } \partial \mathcal{F}_D  $$

The forces on the boundaries need to equal an eventual external force $ \bold{f}$. These are are enforced on the Neumann boundaries:
$$ \sigma \cdot \bold{n} = f \text{   on   } \partial \mathcal{F}_N    $$







