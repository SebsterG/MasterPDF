\section{Discretization of monolithic Fluid-Structure Interaction equations}\label{Discretization}
The temporal discretization is performed using a finite difference scheme and the spatial discretization is treated with the finite element method, following the ideas and notations of \cite{Wick2011}.

The scheme is first introduced in a weak form and using, for simplicity, the harmonic lifting operator. 

\textbf{The full monolithic FSI weak formulations reads}:\\
In the domain $\Omega \in \mathbb{R}^D (D=1,2,3)$ and time interval [0,T],
find $\bold{u} \in \Omega \times \mathbb{R}^+ \rightarrow \mathbb{R}^D$ , $ p \in \Omega \times \mathbb{R}^+ \rightarrow \mathbb{R}$ and $\bold{d} \in \Omega \times \mathbb{R}^+ \rightarrow \mathbb{R}^D$. Let $\phi, \psi$ and $\gamma$ be the test functions used in the weak formulation, which are continuous across the entire domain.
\begin{align}
   (J \rho_f \partial_t \bold{u} , \phi ) + (J (\nabla \bold{u})F^{-1}(\bold{u} - \partial_t d) , \phi )_{\mathcal{\hat{F}}} =& 0 \label{eq:firsweak} \\
   ( J \sigma_{f} F^{-T} , \nabla \phi )_{\mathcal{\hat{F}}} =& 0\\
   (\rho_s \partial_t \bold{u} , \phi)_{\mathcal{\hat{S}}}   + \big(P, \nabla \phi \big)_{\mathcal{\hat{S}}} =& 0\\
   ( \alpha_u \nabla \bold{u}, \nabla \psi)_{\mathcal{\hat{F}}} + (\nabla \cdot (J F^{-1} \bold{u}),\gamma )_{\mathcal{\hat{F}}} =& 0\\
    \delta\big((\partial_t \bold{d} , \psi)_{\mathcal{\hat{S}}}  - ( \bold{u} , \psi )_{\mathcal{\hat{S}}}\big)=&0       \label{eq:condition} \\ 
    \big( J \sigma_{f,p} F^{-T}, \nabla \phi  \big) =& 0 \label{eq:weak} 	 		
\end{align}

Introducing the \textit{$\theta$-scheme} from \cite{Wick2011}, which has the advantage of easily being changed from a backward (implicit), forward(explicit), or a Crank-Nicholson (implicit) scheme, by changing the value of $\theta$. The Crank-Nicholson scheme is of second order, but suffers from instabilities in this monolithic scheme for certain time step values \cite{Wick2011}. A remedy for these instabilities is to chose a Crank-Nicholson scheme that is shifted towards the implicit side. How this is performed will become evident once the scheme is defined.

The variational form is defined by dividing the equations \eqref{eq:firsweak} - \eqref{eq:weak}  into four categories. The four divided categories consists of: a time group $A_T$ (with time derivatives), implicit $A_I$ (terms always kept implicit), pressure $A_P$ and the rest $A_E$ (stress terms and convection):

\begin{align}
A_T(U) &= (J \rho_f \partial_t \bold{u} , \phi ) - (J (\nabla \bold{u})F^{-1}(\partial_t \bold{d}) , \phi )_{\mathcal{\hat{F}}} \\
	    & + (\rho_s \partial_t \bold{u} , \phi)_{\mathcal{\hat{S}}} + (\partial_t \bold{d} , \psi)_{\mathcal{\hat{S}}}  \\
A_I(U) &= ( \alpha_u \nabla \bold{u}, \nabla \psi)_{\mathcal{\hat{F}}} + (\nabla \cdot (J F^{-1} \bold{u}), \gamma)_{\mathcal{\hat{F}}} \\
A_E(U) &= (J (\nabla u)F^{-1} \bold{u} , \phi )_{\mathcal{\hat{F}}} + ( J \sigma_{f,u} F^{-T} , \nabla \phi )_{\mathcal{\hat{F}}} \\
	    & + \big(P, \nabla \phi \big)_{\mathcal{\hat{S}}} - ( \bold{u} , \psi )_{\mathcal{\hat{S}}} \\
A_P(U) &= \big( J \sigma_{f,p} F^{-T}, \nabla \phi  \big)  	 		
\end{align}

Notice that the fluid stress tensors have been split into a velocity and pressure part. 
\begin{align}
\sigma_{f,u} &= \mu ( \nabla u F^{-1} + F^{-T} \nabla u) \\
\sigma_{f,p} &= -p I
\end{align}

For the time group, discretization is done in the following way:
\begin{align}
A_T(U^{n,k}) \approx & \frac{1}{k} \big( \rho_f J^{n,\theta} (\bold{u}^n - \bold{u}^{n-1}) , \phi  \big)_{\mathcal{\hat{F}}} - \frac{1}{k} \big( \rho_f (\nabla u ) (\bold{d}^n - \bold{d}^{n-1}) , \phi \big)_{\mathcal{\hat{F}}} \\
+ & \frac{1}{k} \big( \rho_s  (\bold{u}^n - \bold{u}^{n-1}) , \phi  \big)_{\mathcal{\hat{S}}} +  \frac{1}{k} \big( (\bold{d}^n - \bold{d}^{n-1}) , \psi  \big)_{\mathcal{\hat{S}}}
\end{align}

The Jacobian $J^{n, \theta}$ is expressed as:
\begin{equation}
J^{n, \theta} = \theta J^n + (1-\theta)J^{n-1}
\end{equation}

Let the \textit{$\theta$ scheme} be defined as: 

\begin{align}
& A_T(U^{n,k}) + \theta A_E(U^{n}) + A_P(U^{n}) + A_I(U^{n}) = \\
& - (1-\theta) A_E(U^{n-1}) + \theta \hat{f}^n + (1-\theta) \hat{f}^{n-1}  
\end{align}

By choosing a value of $ \theta = 1$ we obtain the backward Euler scheme, for $ \theta = \frac{1}{2}$ we get the Crank-Nicholson scheme and for the shifted Crank-Nicholson we set $ \theta = \frac{1}{2} + k$, effectively shifting the scheme towards the implicit side. $\hat{f}$ is the body forces. The impact of choosing values for $\theta$ will be investigated in chapter \ref{chap:VV}.





