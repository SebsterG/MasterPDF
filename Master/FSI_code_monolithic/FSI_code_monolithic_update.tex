\begin{comment}
\lstdefinelanguage{Python}{
 keywords={typeof, null, catch, switch, in, int, str, float, self},
 ndkeywords={boolean, throw, import},
 ndkeywords={return, class, if ,elif, endif, while, do, else, True, False , catch, def},
 ndkeywordstyle=\color{blue}\bfseries,
 identifierstyle=\color{black},
 sensitive=false,
 comment=[l]{\#},
 morecomment=[s]{/*}{*/},
 commentstyle=\color{purple}\ttfamily,
 stringstyle=\color{red}\ttfamily,
 backgroundcolor = \color{lightgray}
}
\end{comment}


\chapter{Implementation of Fluid-Structure Interaction in FEniCS}
The discretization described in chapter \ref{Discretization}, is implemented using FEniCS \cite{FENICS}. FEniCS is a platform used to solve partial differential equations. Code is written in python, using FEniCS to easily run efficient finite element code. \newline

The complete code consist of many hundreds of lines of code, and therefore only the most essential parts are covered. The code that has been added is added so that a reader with a minimal skill set in scientific computing and basic knowledge of the finite element method could implement a version of the code.

\section{Variational Formulation}

The variational form can be written directly into FEniCS. A big advantage in FEniCS is that there is a small \say{gap} between mathematical notation and FEniCS syntax. For instance the gradient and divergence, is in FEniCS as written as grad and div respectively. For this reason the basic parts of the code should be self evident.

The code structure has a main script named monolithic.py which from command line takes in arguments specifying the problem to be solved, the version of the variational form and the Newtonsolver. The main script gather the called parts from specific folders. The problem folder contains the specific problem, which contains the necessary boundary conditions, mesh, parameters for fluid an structure, and saves the solutions and other data in a post processing function. The variational form folder contains the fluid and solid variational form, which given in the command line takes a value for $\theta$ and the chosen lifting operator. The Newtonsolver folder contains different versions of the Newtonsolver, which determines which one of the speedups, outlined in chapter \ref{runtime}, to be implemented.

The main script monolithic.py creates the functions, functionspaces and vectorfunction spaces needed. The main script contains the time loop which iterates in time calling the variational form, the newton solver and updating the functions in time. 

The vector functions and functions such as the displacement, velocity, and pressure, are written with a script \say{n} meaning at which time the function is valued. 
\begin{python}\caption{Deformation, velocity and pressure function described at three different timesteps }
d_["n"]  """deformation in the current timestep """
u_["n-1"] """velocity in the last timestep """
p_["n-2"] """pressure in the second last timestep """
\end{python}

We define the linear and nonlinear parts of the fluid and solid variational. All the linear and nonlinear parts are added together to create the full variational form.

psi, phi and gamma are the test functions: $\psi, \phi$ and $\gamma$.
\newpage
\begin{python}
J_theta = theta*J_(d_["n"]) + (1 - theta)*J_(d_["n-1"]) 

F_fluid_linear = rho_f/k*inner(J_theta*(v_["n"] - v_["n-1"]), psi)*dx_f

F_fluid_nonlinear =  Constant(theta)*rho_f*\
inner(J_(d_["n"])*grad(v_["n"])*inv(F_(d_["n"]))*v_["n"], psi)*dx_f

F_fluid_nonlinear += inner(J_(d_["n"])*sigma_f_p(p_["n"], d_["n"])*\
inv(F_(d_["n"])).T, grad(psi))*dx_f

F_fluid_nonlinear += Constant(theta)*inner(J_(d_["n"])\
*sigma_f_u(v_["n"], d_["n"], mu_f)*inv(F_(d_["n"])).T, grad(psi))*dx_f

F_fluid_nonlinear += Constant(1 - theta)*inner(J_(d_["n-1"])*\
sigma_f_u(v_["n-1"], d_["n-1"], mu_f)*inv(F_(d_["n-1"])).T, grad(psi))*dx_f

F_fluid_nonlinear += \
inner(div(J_(d_["n"])*inv(F_(d_["n"]))*v_["n"]), gamma)*dx_f

F_fluid_nonlinear += Constant(1 - theta)*rho_f*\
inner(J_(d_["n-1"])*grad(v_["n-1"])*inv(F_(d_["n-1"]))*v_["n-1"], psi)*dx_f

F_fluid_nonlinear -= rho_f*inner(J_(d_["n"])*\
grad(v_["n"])*inv(F_(d_["n"]))*((d_["n"]-d_["n-1"])/k), psi)*dx_f

delta = 1E10
F_solid_linear = rho_s/k*inner(v_["n"] - v_["n-1"], psi)*dx_s +\
delta*(1/k)*inner(d_["n"] - d_["n-1"], phi)*dx_s -\
delta*inner(Constant(theta)*v_["n"] + Constant(1-theta)*v_["n-1"], phi)*dx_s

F_solid_nonlinear = inner(Piola1(Constant(theta)*d_["n"] +\
Constant(1 - theta)*d_["n-1"], lamda_s, mu_s), grad(psi))*dx_s
\end{python}


\section{Newtons Method Implementation for Solving Fluid-Structure Interaction in FEniCS}
To handle the non-linearities in the scheme we use a Newton solver. FEniCS already has a built-in Newtonsolver, however this solver was not able to compute the monolithic FSI problem because of the pressure function, with our choice of solid stress tensor, only being defined in the fluid domain. We had to manipulate the matrix to ensure that the pressure was zero in the solid domain. This gave the need to implement our own Newtonsolver taking ideas from Mikael Mortensens course in CFD named MEK4300 at the University of Oslo \cite{White2006}.
Following is a print out of the full Newtonsolver, that is called for each time iteration.
\newpage
\begin{python}
def newtonsolver(F, J_nonlinear, A_pre, A, b, bcs, \
                dvp_, up_sol, dvp_res, rtol, atol, max_it, T, t, **monolithic):
    Iter      = 0  """ Setting initial values """
    residual   = 1  
    rel_res    = residual
    lmbda = 1 """  """
    while rel_res > rtol and residual > atol and Iter < max_it:
        if Iter % 10 == 0: """Assembles the Jacobian for each tenth round, in this instance. """
            A = assemble(J_nonlinear, tensor=A, form_compiler_parameters = {"quadrature_degree": 4}) """ Assembles the Jacobian with reduction of the quadrature """
            A.axpy(1.0, A_pre, True)
            A.ident_zeros()  """ Sets values of zero to 1 to ensure zero pressure in the solid domain """

        b = assemble(-F, tensor=b) """ assembling the residual  """

        [bc.apply(A, b, dvp_["n"].vector()) for bc in bcs] """ Applies boundary conditions to the mixed function dvp """
        up_sol.solve(A, dvp_res.vector(), b) """ Solves the matrix equation A * dvp = b """
        dvp_["n"].vector().axpy(lmbda, dvp_res.vector())   """ A fast   """
        [bc.apply(dvp_["n"].vector()) for bc in bcs] 
        rel_res = norm(dvp_res, 'l2')
        residual = b.norm('l2')                     
        if isnan(rel_res) or isnan(residual):
            print "type rel_res: ",type(rel_res)
            t = T*T

        if MPI.rank(mpi_comm_world()) == 0:   """ Prints only out the numeber 0 process when running code in paralell  """
            print "Newton iteration %d: r (atol) = %.3e (tol = %.3e), r (rel) = %.3e (tol = %.3e) " \
        % (Iter, residual, atol, rel_res, rtol)
        Iter += 1

    return dict(t=t)
\end{python}




	