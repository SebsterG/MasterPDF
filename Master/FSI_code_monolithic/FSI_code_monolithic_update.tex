\chapter{Implementation in FENiCS}
Here we will look at the implementation of the monolithic FSI Code in FENiCS. 



\lstdefinelanguage{Python}{
 keywords={typeof, null, catch, switch, in, int, str, float, self},
 ndkeywords={boolean, throw, import},
 ndkeywords={return, class, if ,elif, endif, while, do, else, True, False , catch, def},
 ndkeywordstyle=\color{blue}\bfseries,
 identifierstyle=\color{black},
 sensitive=false,
 comment=[l]{\#},
 morecomment=[s]{/*}{*/},
 commentstyle=\color{purple}\ttfamily,
 stringstyle=\color{red}\ttfamily,
 backgroundcolor = \color{lightgray}
}


All the mappings and stresstensors are made using functions, I will just show some to understand the code later on:

\begin{lstlisting}[language=Python]
def F_(U):
	return (Identity(len(U)) + grad(U))

def J_(U):
	return det(F_(U))

def sigma_f_new(u,p,d,mu_f):
	return -p*Identity(len(u)) + mu_f*(grad(u)*inv(F_(d)) + inv(F_(d)).T*grad(u).T)

\end{lstlisting}

The variational form can be written directly into FEniCS. We write all the forms and add them together to make one big form to be calculated in the upcoming timeloop. We start by looking at the fluid variational form
\begin{lstlisting}[language=Python]
F_fluid = (rho_f/k)*inner(J_(d_["n"])*(v_["n"] - v_["n-1"]), phi)*dx_f
F_fluid += rho_f*inner(J_(d_["n"])*grad(v_["n"])*inv(F_(d_["n"]))*(v_["n"] - (d_["n"]-d_["n-1"])/k), phi)*dx_f
F_fluid += inner(J_(d_["n"])*sigma_f_new(v_["n"], p_["n"], d_["n"], mu_f)*inv(F_(d_["n"])).T, grad(phi))*dx_f
F_fluid -= inner(div(J_(d_["n"])*inv(F_(d_["n"]))*v_["n"]), gamma)*dx_f
\end{lstlisting}

where $dx_f$ is the fluid domain.
Next is the solid variational form:
\begin{lstlisting}[language=Python]
delta = 1E10
	F_solid = rho_s/k*inner(v_["n"] - v_["n-1"], phi)*dx_s
	
	F_solid += inner(Piola1(0.5*(d_["n"] + d_["n-1"]), lamda_s, mu_s), grad(phi))*dx_s

	F_solid += delta*((1./k)*inner(d_["n"] - d_["n-1"],phi)*dx_s - inner(0.5*(v_["n"] + v_["n-1"]), psi)*dx_s)
	
	F_solid += inner(grad(d_["n"]), grad(phi))*dx_f + (1./k)*inner(d_["n"] - d_["n-1"], psi)*dx_f
\end{lstlisting}

The condition $ u = \frac{\partial d}{\partial t} in \mathcal{S} $, is weighted with a delta value. This is done since we compute everything together this is an important condition? vet ikke hva jeg skal skrive her...
We have used a Crank-Nicholson scheme in the solid to preserve the energy, which will be discussed later in the Verification and Validation chapter.

To solve a non-linear problem we need make a newton solver, taken from [Mikael kompendium]. F is derivated wrt to $dvp$ and is assembled to a matrix. -F is assembled as b and we solve until the residual is smaller than a give tolerance. There is also an if test which only assembles the Jacobian the first and tenth time. This reuses the Jacobian to improve speed, as we shall see later. Lastly the the mpi line is when the code is running in parallell that we only print out the values for one of the computational nodes.  
\begin{lstlisting}[language=Python]
    Iter      = 0
    residual   = 1
    rel_res    = residual
    chi = TrialFunction(DVP)
    Jac = derivative(F, dvp_["n"], chi)

    while rel_res > rtol and residual > atol and Iter < max_it:
        if Iter == 0 or Iter == 10:
            A = assemble(Jac, tensor=A)#, keep_diagonal = True)
            A.ident_zeros()
            
        b = assemble(-F)
        
        [bc.apply(A, b, dvp_["n"].vector()) for bc in bcs]
        solve(A, dvp_res.vector(), b)
                   

        dvp_["n"].vector()[:] = dvp_["n"].vector()[:] + lmbda*dvp_res.vector()[:]
        [bc.apply(dvp_["n"].vector()) for bc in bcs]
        
        rel_res = norm(dvp_res, 'l2')
        residual = b.norm('l2')

        if MPI.rank(mpi_comm_world()) == 0:
            print "Newton iteration %d: r (atol) = %.3e (tol = %.3e), r (rel) = %.3e (tol = %.3e) " \
        % (Iter, residual, atol, rel_res, rtol)
        Iter += 1
\end{lstlisting}


In the time loop we call on the solver and update the functions $ v,d,p$ for each round. The counter value is used when we only want to take out values every certain number of time iterations.

\begin{lstlisting}[language=Python]

while t <= T:
    print "Time t = %.5f" % t
    time_list.append(t)
    if t < 2:
        inlet.t = t;
    if t >= 2:
        inlet.t = 2;

    #Reset counters
    atol = 1e-6;rtol = 1e-6; max_it = 100; lmbda = 1.0;

    dvp = Newton_manual(F, dvp, bcs, atol, rtol, max_it, lmbda,dvp_res,VVQ)


    times = ["n-2", "n-1", "n"]
    for i, t_tmp in enumerate(times[:-1]):
   	dvp_[t_tmp].vector().zero()
    	dvp_[t_tmp].vector().axpy(1, dvp_[times[i+1]].vector())

    t += dt
    counter +=1
\end{lstlisting}


	









